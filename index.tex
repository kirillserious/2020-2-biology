\documentclass[a4paper, 11pt]{article}


\usepackage{amsmath}
\usepackage{amssymb}
\usepackage{hyperref}
\usepackage{url}
\usepackage{a4wide}
\usepackage[utf8]{inputenc}
\usepackage[main = russian, english]{babel}
\usepackage[pdftex]{graphicx}
\usepackage{float}
\usepackage{subcaption}
\usepackage{indentfirst}

%% TIKZ
\usepackage{tikz}                % Векторная графика
                                 %
\usepackage{pgfplots}            % % Нужно для вставки графиков из 
\pgfplotsset{compat=newest}      % % matlab2tikz
\usetikzlibrary{plotmarks}       % %
\usetikzlibrary{arrows.meta}     % %
\usepgfplotslibrary{patchplots}  % %
\usepackage{grffile}             % %

\newcommand{\R}{\mathbb{R}}
\newcommand{\N}{\mathbb{N}}

\begin{document}
	\section{Постановка задачи}
	Рассматривается двумерная динамическая система
	\begin{equation}\label{eq:basic-task}
		\left\{\begin{aligned}
			\dot u &= \frac{au(K - u)}{K} - \frac{buv}{1+Au} + d_1 u_{xx} \\
			\dot v &= -cv + \frac{duv}{1 + Au} +d_2 v_{xx},
		\end{aligned}\right.
	\end{equation}
	где $a$, $b$, $c$, $d$, $A$, $K$, $d_1$, $d_2$~--- положительные параметры, $(u,\,v) \in \R^2_{+}$.
	Здесь $u(x,t)$ ($v(x,t))$~--- плотность популяции жертв (хищников) в точке с координатой $x$ в момент времени $t$, $t > 0$, $x \in \R$, $d_1,\,d_2$~--- величины коэффициентов диффузии жертв и хищников соответственно.

	В рамках задачи, для системы \eqref{eq:basic-task} требуется:
	\begin{enumerate}
		\item Уменьшить число параметров, сделав замену переменных;
		\item Исследовать фазовый портрет нераспределённой системы;
		\item Проверить наличие решений распределённой системы;
		\item Выбрать подходящее решение и скорость волны;
		\item Привести иллюстрации и графики фазовых портретов и решений.
	\end{enumerate}


	\section{Уменьшение числа параметров}
	Для уменьшения числа параметров проведём линейную замену переменных, которая не изменит свойств исходной системы. Итак, пусть
	$$
		\tilde u = \alpha u, \;
		\tilde v = \beta v,  \;
		\tilde t = \gamma t, \;
		\tilde x = \delta x.
	$$
	Тогда частные производные исходных переменных $u$ и $v$ примут вид:
	$$
		u_t = \frac{\partial}{\partial t} \left(\frac{\tilde u}{\alpha}\right) =
		\frac{\gamma}{\alpha}\tilde u_{\tilde t},
		\quad
		u_x = \frac{\partial}{\partial x} \left(\frac{\tilde u}{\alpha}\right) =
		\frac{\delta}{\alpha}\tilde u_{\tilde x},
		\quad
		u_{xx} = \frac{\delta^2}{\alpha}\tilde u_{\tilde x \tilde x},
	$$
	$$
		v_t = \frac{\partial}{\partial t} \left(\frac{\tilde v}{\beta}\right) =
		\frac{\gamma}{\beta}\tilde v_{\tilde t},
		\quad
		v_x = \frac{\partial}{\partial x} \left(\frac{\tilde v}{\beta}\right) =
		\frac{\delta}{\beta}\tilde v_{\tilde x},
		\quad
		v_{xx} = \frac{\delta^2}{\beta}\tilde v_{\tilde x \tilde x}.
	$$
	Подстановкой в рассматриваемую систему \eqref{eq:basic-task} получим
	\begin{equation}\label{eq:changed-system}
		\left\{\begin{aligned}
			\frac{\gamma}{\alpha} \tilde u_{\tilde t} =&
			\frac{a}{\alpha K}\tilde u(K - \tilde u) -
			\frac{b}{\alpha\beta}\cdot\frac{\tilde u \tilde v}{1 + \frac{A}{\alpha}\tilde u} +
			\frac{d_1 \delta^2}{\alpha} \tilde u_{\tilde x \tilde x}\\
			\frac{\gamma}{\beta} \tilde v_{\tilde t} =&
			-\frac{c}{\beta} \tilde v +
			\frac{d}{\alpha\beta}\cdot\frac{\tilde u \tilde v}{1 + \frac{A}{\alpha}\tilde u} +
			\frac{d_2 \delta^2}{\beta} \tilde v_{\tilde x \tilde x}.
		\end{aligned}\right.
	\end{equation}

	Заметим, что если выбрать коэффициенты замены таким образом:
	$$
		\alpha = A,\; \beta = \frac{Ab}{d},\;
		\gamma = \frac{d}{A}, \delta^2 = \frac{d}{d_1 A},
	$$
	то система \eqref{eq:changed-system} примет вид
	\begin{equation}
		\left\{\begin{aligned}
		\tilde u_{\tilde t} =& \frac{aA}{d} \tilde u -
		\frac{a}{dK} \tilde u^2 -
		\frac{\tilde u \tilde v}{1 + \tilde u} -
		\tilde u_{\tilde x \tilde x}
		\\
		\tilde v_{\tilde t} =& - \frac{cA}{d}\tilde v +
		\frac{\tilde u \tilde v}{1 + \tilde u} +
		\frac{d_2}{d_1} \tilde v_{\tilde x \tilde x}.
		\end{aligned}\right.
	\end{equation}

	Таким образом, всюду далее будем исследовать следующую эквивалентную исходной систему с положительными параметрами $A$, $B$, $C$ и $D$:
	\begin{equation}\label{eq:main-task}
		\left\{\begin{aligned}
		\dot u &= -A u^2 + B u - \frac{uv}{1+u} - u_{xx}\\
		\dot v &= -C v + \frac{uv}{1+u} + Dv_{xx}.
		\end{aligned}\right.
	\end{equation}


	\section{Исследование сосредоточенной системы}
	Мы будем понимать под \textit{сосредоточенной} систему, лишённую диффузии.
	Для нашей задачи \eqref{eq:main-task} это соответственно:
	\begin{equation}\label{eq:simple-system}
		\left\{\begin{aligned}
		\dot u &= -A u^2 + Bu - \frac{uv}{1 + u}\\
		\dot v &= -C v + \frac{uv}{1 + u}.
		\end{aligned}\right.
	\end{equation}
	Система \eqref{eq:simple-system} не зависит от координаты $x$, поэтому в данном разделе рассматриваем $u(x,t) \equiv u(t)$, $v(x, t) \equiv v(t)$.

	\subsection{Поиск неподвижных точек}
	Найдем неподвижные точки системы \eqref{eq:simple-system}:
	$$
		\left\{\begin{aligned}
		0 &= -A u^2 + Bu - \frac{uv}{1 + u}\\
		0 &= -C v + \frac{uv}{1 + u}.
		\end{aligned}\right.
		\Longleftrightarrow
		\left\{\begin{aligned}
		0 &= u\left(-A u + B - \frac{v}{1 + u}\right)\\
		0 &= v\left(-C + \frac{u}{1 + u}\right).
		\end{aligned}\right.
		\Longleftrightarrow
	$$
	$$
		\Longleftrightarrow
		\left\{\begin{aligned}
		&\left[\begin{aligned}
			0 &= u \\
			0 &= -Au + B -\frac{v}{1 + u}
		\end{aligned}\right.
		\\
		&\left[\begin{aligned}
			0 &= v \\
			0 &= -C + \frac{u}{1 + u}
		\end{aligned}\right.
		\end{aligned}\right.
		\Longleftrightarrow
		\left[\begin{aligned}
		&\left\{\begin{aligned}
			u &= 0 \\
			v &=0
		\end{aligned}\right.
		\\
		&\left\{\begin{aligned}
			u &= \frac{B}{A} \\
			v &= 0
		\end{aligned}\right.
		\\
		&\left\{\begin{aligned}
			u &= \frac{C}{1 - C} \\
			v &= \frac{B}{1-C} - \frac{AC}{(1 - C)^2}.
		\end{aligned}\right.
		\end{aligned}\right.
	$$
	Таким образом, неподвижными для нашей системы являются точки
	$$
		P_1 = (0,\,0),
		\quad
		P_2 = \left(\frac{B}{A},\,0\right)
		\quad \mbox{и} \quad 
		P_3 = \left(\frac{C}{1 - C},\, \frac{B}{1 - C} - \frac{AC}{(1 - C)^2}\right).
	$$
	Отметим также, что точка $P_3$ лежит в первой координатной четверти, если выполнено условие
	\begin{equation}
		C \leqslant \frac{B}{A + B} < 1,
	\end{equation}
        и при равенстве $C = \frac{B}{A+B}$ совпадает с точкой $P_2$.

	\subsection{Типы неподвижных точек}
	Определим типы неподвижных точек.
	Для этого приведём якобиан системы \eqref{eq:simple-system}:
	\begin{equation}
		J(u, v) =
		\begin{pmatrix}
			-2Au+B-\frac{v}{(u+1)^2} & -\frac{u}{1 + u} \\
			\frac{v}{(u+1)^2}       & -C +\frac{u}{1 + u}
		\end{pmatrix}.
	\end{equation}

	Найдём собственные значения матрицы Якоби $J(u,v)$ для точки $P_1 = (0,\,0)$. Выпишем характеристический многочлен:
	$$
		\chi_{1}(\lambda)
		=
		\mathrm{det} \begin{pmatrix}
			B - \lambda & 0 \\
			0           & -C -\lambda
		\end{pmatrix}
		=
		(B - \lambda)(-C - \lambda).
	$$
	Приравняв нулю характеристический многочлен $\chi_1(\lambda)$, получили собственные значения:
	$$
	       \lambda_1 = B, \quad \lambda_2 = -C.	
	$$
	Заметим, что при любом значении параметров собственные числа $\lambda_1$ и $\lambda_2$ вещественные, и $\lambda_1 > 0$, а $\lambda_2 < 0$, то есть точка $P_1$ является седлом.

        \begin{figure}[h]
                \centering
                % This file was created by matlab2tikz.
%
%The latest updates can be retrieved from
%  http://www.mathworks.com/matlabcentral/fileexchange/22022-matlab2tikz-matlab2tikz
%where you can also make suggestions and rate matlab2tikz.
%
\definecolor{mycolor1}{rgb}{0.00000,0.44700,0.74100}%
\definecolor{mycolor2}{rgb}{0.85000,0.32500,0.09800}%
\definecolor{mycolor3}{rgb}{0.92900,0.69400,0.12500}%
\definecolor{mycolor4}{rgb}{0.49400,0.18400,0.55600}%
\definecolor{mycolor5}{rgb}{0.46600,0.67400,0.18800}%
\definecolor{mycolor6}{rgb}{0.30100,0.74500,0.93300}%
\definecolor{mycolor7}{rgb}{0.63500,0.07800,0.18400}%
%
\begin{tikzpicture}

\begin{axis}[%
width=4.596in,
height=3.502in,
at={(0.771in,0.473in)},
scale only axis,
xmin=0,
xmax=0.0707106781186548,
xlabel style={font=\color{white!15!black}},
xlabel={u},
ymin=0,
ymax=0.0707106781186548,
ylabel style={font=\color{white!15!black}},
ylabel={v},
axis background/.style={fill=white},
axis x line*=bottom,
axis y line*=left,
xmajorgrids,
ymajorgrids,
legend style={legend cell align=left, align=left, draw=white!15!black}
]
\addplot[-Straight Barb, color=mycolor1, point meta={sqrt((\thisrow{u})^2+(\thisrow{v})^2)}, point meta min=0, quiver={u=\thisrow{u}, v=\thisrow{v}, every arrow/.append style={-{Straight Barb[angle'=18.263, scale={10/1000*\pgfplotspointmetatransformed}]}}}]
 table[row sep=crcr] {%
x	y	u	v\\
0.00634239196565645	0.0997986676471884	0.000294178009136761	-0.00488912666390956\\
0.00663656997479321	0.0949095409832789	0.000309389453627114	-0.00464824393514479\\
0.00694595942842033	0.0902612970481341	0.000325360794920581	-0.00441921770889496\\
0.00727132022334091	0.0858420793392391	0.000342130525494518	-0.00420149523669107\\
0.00761345074883542	0.0816405841025481	0.000923396259671196	-0.00952219594237623\\
0.00853684700850662	0.0721183881601718	0.00104487676247542	-0.00840351352461931\\
0.00958172377098205	0.0637148746355525	0.0011818505317258	-0.00741493645398715\\
0.0107635743027078	0.0562999381815654	0.0013362912590174	-0.00654423010857997\\
0.0120998655617252	0.0497557080729854	0.00151029884015363	-0.00577756778843826\\
0.0136101644018789	0.0439781402845472	0.00170606543966348	-0.00509886826814924\\
0.0153162298415424	0.0388792720163979	0.00192627349915185	-0.00449913811428972\\
0.0172425033406942	0.0343801339021082	0.00217386433388206	-0.00397090592119138\\
0.0194163676745763	0.0304092279809168	0.00245192083844625	-0.00350576818368029\\
0.0218682885130225	0.0269034597972365	0.00276360102500811	-0.00309404553886898\\
0.0246318895380306	0.0238094142583675	0.003112651289573	-0.0027302649199213\\
0.0277445408276036	0.0210791493384462	0.00350313069623177	-0.00240984854508396\\
0.0312476715238354	0.0186693007933623	0.00393918676453932	-0.00212768562017078\\
0.0351868582883747	0.0165416151731915	0.00442493563462976	-0.00187797550373159\\
0.0396117939230045	0.0146636396694599	0.00496494015128478	-0.00165738212107097\\
0.0445767340742892	0.0130062575483889	0.00556401097610613	-0.00146308785640037\\
0.0501407450503954	0.0115431696919886	0.00622680757900087	-0.001291982054014\\
0.0563675526293963	0.0102511876379746	0.00695760187936175	-0.00114060902848897\\
0.063325154508758	0.00911057860948561	0.00776036516298564	-0.00100693033600348\\
0.0710855196717436	0.00810364827348213	0.0086389464962808	-0.0008892019165912\\
0.0797244661680244	0.00721444635689093	0.00959644901765683	-0.000785528982188911\\
0.0893209151856813	0.00642891737470202	0.0106346768662197	-0.00069386726501237\\
0.099955592051901	0.00573505010968965	0.0117534067018195	-0.000612966562924475\\
0.111708998753721	0.00512208354676517	0.0129514264296573	-0.00054174046081394\\
0.124660425183378	0.00458034308595123	0.014225807236978	-0.000479033636675616\\
0.138886232420356	0.00410130944927562	0.015570561189172	-0.000423644486206724\\
0.154456793609528	0.00367766496306889	0.0169754678524322	-0.000374801813351943\\
0.17143226146196	0.00330286314971695	0.0184281989492141	-0.000331824412010478\\
0.189860460411174	0.00297103873770647	0.0199139991511163	-0.000294006424245777\\
0.20977445956229	0.0026770323134607	0.021413074867693	-0.000260643329566579\\
0.231187534429983	0.00241638898389412	0.0229014684947516	-0.000231255663867776\\
0.254089002924735	0.00218513332002634	0.0243530867933408	-0.000205413884188801\\
0.278442089718076	0.00197971943583754	0.0257404483123432	-0.000182685242233092\\
0.304182538030419	0.00179703419360445	0.0270318180730013	-0.000162655081884228\\
0.33121435610342	0.00163437911172022	0.028197003878113	-0.000145023495956685\\
0.359411359981533	0.00148935561576354	0.0292063574213747	-0.000129519590845854\\
0.388617717402908	0.00135983602491768	0.0177987424182152	-7.05433451581488e-05\\
0.406416459821123	0.00128929267975953	0.0180488955647126	-6.60584686558918e-05\\
0.424465355385836	0.00122323421110364	0.0182535258016427	-6.19002556953208e-05\\
0.442718881187478	0.00116133395540832	0.0184104724703746	-5.80447488379421e-05\\
};
\addlegendentry{data1}

\addplot[-Straight Barb, color=mycolor2, point meta={sqrt((\thisrow{u})^2+(\thisrow{v})^2)}, point meta min=0, quiver={u=\thisrow{u}, v=\thisrow{v}, every arrow/.append style={-{Straight Barb[angle'=18.263, scale={10/1000*\pgfplotspointmetatransformed}]}}}]
 table[row sep=crcr] {%
x	y	u	v\\
0.0126592453573749	0.0991954812830795	0.000587393313537616	-0.0048589008721229\\
0.0132466386709125	0.0943365804109566	0.000617559675247887	-0.00461821283515319\\
0.0138641983461604	0.0897183675758034	0.000649201748371665	-0.00438943793603411\\
0.0145134000945321	0.0853289296397693	0.000682390510068385	-0.00417201724083514\\
0.0151957906046005	0.0811569123989342	0.00182807600462886	-0.00939547413437358\\
0.0170238666092293	0.0717614382645606	0.0020643289640637	-0.00829250522469302\\
0.0190881955732931	0.0634689330398676	0.00232954540435288	-0.00731782569453421\\
0.0214177409776459	0.0561511073453334	0.00262709404635217	-0.00645925813030297\\
0.0240448350239981	0.0496918492150304	0.00296048281630105	-0.0057031521600645\\
0.0270053178402991	0.0439886970549659	0.00333329759799731	-0.00503387573692798\\
0.0303386154382965	0.0389548213180379	0.00374967616699629	-0.00444252268244653\\
0.0340882916052927	0.0345122986355914	0.00421405348230348	-0.00392161637429966\\
0.0383023450875962	0.0305906822612917	0.00473089092586158	-0.00346284873671043\\
0.0430332360134578	0.0271278335245813	0.00530454209430677	-0.00305686766417438\\
0.0483377781077646	0.0240709658604069	0.0059395937802873	-0.00269823794377453\\
0.0542773718880519	0.0213727279166324	0.00664077285982832	-0.00238234654990927\\
0.0609181447478802	0.0189903813667231	0.00741247049738913	-0.002104133306433\\
0.0683306152452693	0.0168862480602901	0.00825843380665618	-0.00185803939468698\\
0.0765890490519255	0.0150282086656031	0.00918154305636111	-0.00164073852001655\\
0.0857705921082866	0.0133874701455866	0.0101842685389425	-0.00144936762606224\\
0.0959548606472292	0.0119381025195244	0.011267982586139	-0.00128084231695472\\
0.107222843233368	0.0106572602025696	0.0124321817402784	-0.00113188178845813\\
0.119655024973647	0.00952537841411151	0.013673430599736	-0.00100044143068051\\
0.133328455573383	0.00852493698343099	0.0149868437506735	-0.000884732828456937\\
0.148315299324056	0.00764020415497406	0.0163654422259109	-0.000782873557465162\\
0.164680741549967	0.00685733059750889	0.0177983485685026	-0.000692937502455089\\
0.18247909011847	0.0061643930950538	0.01926998952546	-0.00061365947509018\\
0.20174907964393	0.00555073361996363	0.0207624325216946	-0.000543915150490426\\
0.222511512165624	0.0050068184694732	0.0222554414387685	-0.000482552643254346\\
0.244766953604393	0.00452426582621885	0.0237235108568143	-0.00042844124943685\\
0.268490464461207	0.004095824576782	0.0251385124175482	-0.000380791854981797\\
0.293628976878755	0.00371503272180021	0.0264708963403413	-0.00033889458812244\\
0.320099873219096	0.00337613813367777	0.0276908520194167	-0.000302042165788565\\
0.347790725238513	0.0030740959678892	0.0287662652223187	-0.000269566970260912\\
0.376556990460832	0.00280452899762829	0.0296699419934996	-0.000240975350013189\\
0.406226932454332	0.0025635536476151	0.0303768355201265	-0.000215821009162974\\
0.436603767974458	0.00234773263845213	0.030865271747712	-0.000193670202942702\\
0.46746903972217	0.00215406243550942	0.0311192895514613	-0.000174126804136824\\
0.498588329273631	0.0019799356313726	0.0311335926902255	-0.000156885556532882\\
0.529721921963857	0.00182305007483972	0.0309079083325191	-0.000141671506183623\\
0.560629830296376	0.0016813785686561	0.018125206306839	-7.75552919053723e-05\\
0.578755036603215	0.00160382327675072	0.0179160254471467	-7.31572809019175e-05\\
0.596671062050362	0.00153066599584881	0.0176623652750898	-6.90639202886255e-05\\
0.614333427325451	0.00146160207556018	0.0173668445868541	-6.52522608869523e-05\\
};
\addlegendentry{data2}

\addplot[-Straight Barb, color=mycolor3, point meta={sqrt((\thisrow{u})^2+(\thisrow{v})^2)}, point meta min=0, quiver={u=\thisrow{u}, v=\thisrow{v}, every arrow/.append style={-{Straight Barb[angle'=18.263, scale={10/1000*\pgfplotspointmetatransformed}]}}}]
 table[row sep=crcr] {%
x	y	u	v\\
0.018925124436041	0.0981928697262707	0.000878772528708407	-0.00480913349409587\\
0.0198038969647494	0.0933837362321748	0.000923584187523818	-0.0045696547026428\\
0.0207274811522732	0.088814081529532	0.000970540895202869	-0.00434209543196307\\
0.0216980220474761	0.0844719860975689	0.00101974009434557	-0.00412589296479811\\
0.0227177621418217	0.0803460931327708	0.00271177901895561	-0.00923362539478914\\
0.0254295411607773	0.0711124677379817	0.0030559858195978	-0.00815056583570674\\
0.0284855269803751	0.0629619019022749	0.00344070139811203	-0.00719350248169966\\
0.0319262283784871	0.0557683994205753	0.00387017630805076	-0.00635037349153391\\
0.0357964046865379	0.0494180259290414	0.00434872721505387	-0.00560776073869725\\
0.0401451319015918	0.0438102651903441	0.00488064052901527	-0.00495051424607186\\
0.045025772430607	0.0388597509442723	0.00547051292355492	-0.00436985241668778\\
0.050496285354162	0.0344898985275845	0.00612312069087095	-0.00385833744521471\\
0.0566194060450329	0.0306315610823698	0.00684299873958319	-0.00340779411795058\\
0.0634624047846161	0.0272237669644192	0.00763419489621592	-0.00300920164416725\\
0.071096599680832	0.024214565320252	0.0085001516485395	-0.00265718721034958\\
0.0795967513293715	0.0215573781099024	0.00944403068185992	-0.00234713844602372\\
0.0890407820112314	0.0192102396638787	0.0104680660681213	-0.00207407108042645\\
0.0995088480793528	0.0171361685834522	0.0115729215365201	-0.0018326519737938\\
0.111081769615873	0.0153035166096584	0.0127567477066387	-0.00161958081448958\\
0.123838517322512	0.0136839357951688	0.014016453329076	-0.00143197928963108\\
0.137854970651588	0.0122519565055378	0.0153470241621934	-0.00126680535972357\\
0.153201994813781	0.0109851511458142	0.0167399616896984	-0.00112092446540567\\
0.169941956503479	0.00986422668040851	0.0181822408694162	-0.000992296781334379\\
0.188124197372896	0.00887192989907414	0.0196585662498583	-0.000879114506849733\\
0.207782763622754	0.0079928153922244	0.021151252871122	-0.000779516705883762\\
0.228934016493876	0.00721329868634064	0.0226374242702815	-0.000691663308516246\\
0.251571440764157	0.00652163537782439	0.0240907651624476	-0.000614283521438406\\
0.275662205926605	0.00590735185638599	0.0254831549813367	-0.000546236219421379\\
0.301145360907942	0.00536111563696461	0.0267856582489333	-0.000486378989618496\\
0.327931019156875	0.00487473664734611	0.027966055914965	-0.000433626287702093\\
0.35589707507184	0.00444111035964402	0.0289954597492451	-0.000387182776949544\\
0.384892534821085	0.00405392758269448	0.0298463714073021	-0.000346329612715756\\
0.414738906228387	0.00370759796997872	0.0304940369624969	-0.000310365325879583\\
0.445232943190884	0.00339723264409914	0.0309178716388194	-0.000278645775330262\\
0.476150814829704	0.00311858686876888	0.0311075091745241	-0.000250677343716591\\
0.507258324004228	0.00286790952505228	0.0310573441802429	-0.000226014854461058\\
0.538315668184471	0.00264189467059123	0.0307666059272179	-0.000204232331528972\\
0.569082274111688	0.00243766233906225	0.0302446520653613	-0.000184948711771033\\
0.59932692617705	0.00225271362729122	0.0295086907475375	-0.000167865459300216\\
0.628835616924587	0.00208484816799101	0.0285800054953158	-0.000152715776897965\\
0.657415622419903	0.00193213239109304	0.0163705013549134	-8.37527678672187e-05\\
0.673786123774816	0.00184837962322582	0.0159576282278477	-7.93685800939837e-05\\
0.689743752002664	0.00176901104313184	0.0155185985161701	-7.5271104672698e-05\\
0.705262350518834	0.00169373993845914	0.0150571461658943	-7.1438949135839e-05\\
};
\addlegendentry{data3}

\addplot[-Straight Barb, color=mycolor4, point meta={sqrt((\thisrow{u})^2+(\thisrow{v})^2)}, point meta min=0, quiver={u=\thisrow{u}, v=\thisrow{v}, every arrow/.append style={-{Straight Barb[angle'=18.263, scale={10/1000*\pgfplotspointmetatransformed}]}}}]
 table[row sep=crcr] {%
x	y	u	v\\
0.0251147987181079	0.0967948701396356	0.00116744367375622	-0.00474003140349043\\
0.0262822423918641	0.0920548387361452	0.00122654107368184	-0.0045027835171475\\
0.027508783465546	0.0875520552189977	0.00128840457150311	-0.00427740985726592\\
0.0287971880370491	0.0832746453617318	0.0013531521274603	-0.0040633464369631\\
0.0301503401645094	0.0792112989247687	0.00357244403306531	-0.00903799253313815\\
0.0337227841975747	0.0701733063916305	0.00401774753377309	-0.00797884943235495\\
0.0377405317313478	0.0621944569592756	0.00451329314050727	-0.00704295275426774\\
0.0422538248718551	0.0551515042050078	0.00506376518672576	-0.00621841223835312\\
0.0473175900585808	0.0489330919666547	0.00567377847855376	-0.00549209440519098\\
0.0529913685371346	0.0434409975614637	0.00634773008859252	-0.00484936153907419\\
0.0593390986257271	0.0385916360223895	0.00708988596821402	-0.0042815920531904\\
0.0664289845939411	0.0343100439691991	0.00790447141222259	-0.00378142793022471\\
0.0743334560061637	0.0305286160389744	0.00879512645332961	-0.00334086079576047\\
0.0831285824594933	0.027187755243214	0.00976449156446771	-0.00295120666910171\\
0.092893074023961	0.0242365485741123	0.0108135917853888	-0.0026071779764393\\
0.10370666580935	0.021629370597673	0.0119426616026184	-0.00230419412458448\\
0.115649327411968	0.0193251764730885	0.0131504401985122	-0.00203736820152854\\
0.128799767610481	0.0172878082715599	0.0144330839630986	-0.00180158311339423\\
0.143232851573579	0.0154862251581657	0.0157829283135743	-0.00159357939493834\\
0.159015779887153	0.0138926457632274	0.0171904006361319	-0.0014104871989319\\
0.176206180523285	0.0124821585642955	0.0186435807702264	-0.00124931954396141\\
0.194849761293512	0.0112328390203341	0.0201259007686915	-0.00110707194798652\\
0.214975662062203	0.0101257670723475	0.0216162654131184	-0.000981720189726072\\
0.236591927475322	0.00914404688262146	0.0230912721101542	-0.000871455500843166\\
0.259683199585476	0.0082725913817783	0.0245257319720452	-0.000774446326978651\\
0.284208931557521	0.00749814505479965	0.0258897005599923	-0.000688924150319185\\
0.310098632117513	0.00680922090448046	0.0271531478222195	-0.000613623056177659\\
0.337251779939733	0.0061955978483028	0.0282858140585731	-0.000547400866038236\\
0.365537593998306	0.00564819698226457	0.0292586129880574	-0.000489131708837342\\
0.394796206986363	0.00515906527342723	0.0300430675900122	-0.000437767722734576\\
0.424839274576376	0.00472129755069265	0.030618714486277	-0.000392518647874558\\
0.455457989062653	0.00432877890281809	0.0309687318452392	-0.000352670396108159\\
0.486426720907892	0.00397610850670993	0.0310807007322336	-0.000317534685453348\\
0.517507421640125	0.00365857382125659	0.0309507449027804	-0.000286489677914759\\
0.548458166542906	0.00337208414334183	0.0305850812180205	-0.000259051212619977\\
0.579043247760926	0.00311303293072185	0.029994846241189	-0.000234784679242465\\
0.609038094002115	0.00287824825147938	0.0291952647211893	-0.000213277355391949\\
0.638233358723305	0.00266497089608743	0.0282106437373396	-0.000194163825792111\\
0.666444002460644	0.00247080707029532	0.02706824351727	-0.000177157053725193\\
0.693512245977914	0.00229365001657013	0.0257973694135796	-0.0001620022475579\\
0.719309615391494	0.00213164776901223	0.0145346519509726	-8.88388667462689e-05\\
0.733844267342466	0.00204280890226596	0.0140456423491737	-8.44518511855835e-05\\
0.74788990969164	0.00195835705108038	0.0135450908263879	-8.03369271735496e-05\\
0.761435000518028	0.00187802012390683	0.0130364112135384	-7.64742417473123e-05\\
};
\addlegendentry{data4}

\addplot[-Straight Barb, color=mycolor5, point meta={sqrt((\thisrow{u})^2+(\thisrow{v})^2)}, point meta min=0, quiver={u=\thisrow{u}, v=\thisrow{v}, every arrow/.append style={-{Straight Barb[angle'=18.263, scale={10/1000*\pgfplotspointmetatransformed}]}}}]
 table[row sep=crcr] {%
x	y	u	v\\
0.0312033445698487	0.0950071117740945	0.00145253634602504	-0.00465187880767477\\
0.0326558809158738	0.0903552329664198	0.00152551351180415	-0.00441788546435745\\
0.0341813944276779	0.0859373475020623	0.00160182832401692	-0.00419566819893108\\
0.0357832227516948	0.0817416793031312	0.00168161346496114	-0.00398466436893649\\
0.037464836216656	0.0777570149341947	0.00440819791310558	-0.00880998738999274\\
0.0418730341297615	0.068947027544202	0.00494776854535422	-0.00777857055763211\\
0.0468208026751158	0.0611684569865699	0.00554562704879324	-0.00686721516215714\\
0.052366429723909	0.0543012418244128	0.00620650698194201	-0.00606425601797482\\
0.058572936705851	0.0482369858064379	0.00693489421113433	-0.00535689330077859\\
0.0655078309169854	0.0428800925056593	0.00773480566720328	-0.00473102929014443\\
0.0732426365841886	0.0381490632155149	0.00860956918337681	-0.00417823604268287\\
0.0818522057675654	0.033970827172832	0.00956219151332949	-0.00369127353400845\\
0.0914143972808949	0.0302795536388236	0.0105947285776144	-0.00326233134819384\\
0.102009125858509	0.0270172222906298	0.0117076496358744	-0.00288306887169591\\
0.113716775494384	0.0241341534189339	0.0128988266984785	-0.00254830537270829\\
0.126615602192862	0.0215858480462256	0.0141648440798863	-0.00225351975433388\\
0.140780446272749	0.0193323282918917	0.01550034275913	-0.00199394155832049\\
0.156280789031878	0.0173383867335712	0.0168964474775299	-0.00176466477398889\\
0.173177236509408	0.0155737219595823	0.0183397368141339	-0.00156248516233592\\
0.191516973323542	0.0140112367972464	0.0198145027043754	-0.00138456156437122\\
0.211331476027918	0.0126266752328752	0.0213026760984773	-0.00122797358276407\\
0.232634152126395	0.0113987016501111	0.0227810330065106	-0.00108983850513543\\
0.255415185132906	0.0103088631449757	0.0242230593483098	-0.000968158913087883\\
0.279638244481215	0.00934070423188779	0.0256005016751372	-0.000861140686227045\\
0.305238746156353	0.00847956354566074	0.0268843968778855	-0.000766989810178892\\
0.332123143034238	0.00771257373548185	0.0280427119807909	-0.000684003839189143\\
0.360165855015029	0.00702856989629271	0.0290469861352471	-0.000610932281025411\\
0.389212841150276	0.00641763761526729	0.029870262830217	-0.000546644877776392\\
0.419083103980493	0.0058709927374909	0.0304884327659936	-0.00049003961751668\\
0.449571536746487	0.00538095311997422	0.0308818280435095	-0.000440105463931244\\
0.480453364789996	0.00494084765604298	0.0310409993406132	-0.000396066461504654\\
0.51149436413061	0.00454478119453833	0.0309612702719095	-0.000357222935770451\\
0.542455634402519	0.00418755825876787	0.0306427746951838	-0.000322905704301826\\
0.573098409097703	0.00386465255446605	0.0300957669076837	-0.00029251642560603\\
0.603194176005387	0.00357213612886002	0.0293380501888596	-0.000265586072336756\\
0.632532226194246	0.00330655005652326	0.0283913809089231	-0.000241695600258146\\
0.660923607103169	0.00306485445626512	0.0272801993986143	-0.000220448543220883\\
0.688203806501784	0.00284440591304423	0.026035012716636	-0.000201495636660971\\
0.71423881921842	0.00264291027638326	0.0246856280917929	-0.000184563137427293\\
0.738924447310212	0.00245834713895597	0.0232622811941322	-0.000169409320397791\\
0.762186728504345	0.00228893781855818	0.0129416967921587	-9.27913184902876e-05\\
0.775128425296503	0.00219614650006789	0.0124332021351934	-8.84081913736265e-05\\
0.787561627431697	0.00210773830869426	0.0119229559135967	-8.42845011067546e-05\\
0.799484583345293	0.00202345380758751	0.0114137309672584	-8.04018179712573e-05\\
};
\addlegendentry{data5}

\addplot[-Straight Barb, color=mycolor6, point meta={sqrt((\thisrow{u})^2+(\thisrow{v})^2)}, point meta min=0, quiver={u=\thisrow{u}, v=\thisrow{v}, every arrow/.append style={-{Straight Barb[angle'=18.263, scale={10/1000*\pgfplotspointmetatransformed}]}}}]
 table[row sep=crcr] {%
x	y	u	v\\
0.0371662455660328	0.0928367933016073	0.00173318187724417	-0.00454503605127254\\
0.0388994274432769	0.0882917572503347	0.00181959023474942	-0.0043153176396953\\
0.0407190176780263	0.0839764396106394	0.00190985767997686	-0.00409722339616218\\
0.0426288753580032	0.0798792162144772	0.00200412605106848	-0.00389019472451001\\
0.0446330014090717	0.0759890214899672	0.00521734317895023	-0.00855107873941208\\
0.0498503445880219	0.0674379427505552	0.00584443869881165	-0.00755099454986718\\
0.0556947832868336	0.059886948200688	0.00653631301854889	-0.00666737357688864\\
0.0622310963053825	0.0532195746237993	0.00729742514520076	-0.0058888265959237\\
0.0695285214505832	0.0473307480278756	0.00813178234269445	-0.00520293320044021\\
0.0776603037932777	0.0421278148274354	0.00904262356194881	-0.00459616117507702\\
0.0867029273552265	0.0375316536523584	0.0100318889606535	-0.00406030814586689\\
0.0967348163158799	0.0334713455064915	0.011100890126646	-0.00358828777360434\\
0.107835706442526	0.0298830577328872	0.0122496373977751	-0.00317251630777573\\
0.120085343840301	0.0267105414251115	0.0134759429895834	-0.00280500385022527\\
0.133561286829884	0.0239055375748862	0.0147741860850901	-0.00248069714118446\\
0.148335472914975	0.0214248404337017	0.0161370329805862	-0.00219515977158653\\
0.164472505895561	0.0192296806621152	0.0175549148525973	-0.00194375528320983\\
0.182027420748158	0.0172859253789054	0.0190139954424584	-0.00172178721106846\\
0.201041416190616	0.0155641381678369	0.0204957686717504	-0.00152612069525438\\
0.221537184862367	0.0140380174725825	0.0219793592141009	-0.00135396136398542\\
0.243516544076468	0.0126840561085971	0.0234418427648624	-0.00120246648876953\\
0.26695838684133	0.0114815896198276	0.0248552806408038	-0.00106887190237401\\
0.291813667482134	0.0104127177174536	0.0261903914619055	-0.000951218286215074\\
0.318004058944039	0.00946149943123848	0.027417068763805	-0.000847740060795561\\
0.345421127707844	0.00861375937044292	0.028505709902705	-0.000756689056275609\\
0.37392683761055	0.00785707031416731	0.0294259526887595	-0.000676428117623445\\
0.403352790299309	0.00718064219654387	0.0301540418116498	-0.000605731957602596\\
0.433506832110959	0.00657491023894127	0.030669181256815	-0.000543492982962947\\
0.464176013367774	0.00603141725597833	0.0309545493165415	-0.000488639828779528\\
0.495130562684315	0.0055427774271988	0.0310006673188755	-0.000440199791782512\\
0.526131230003191	0.00510257763541629	0.0308084998503489	-0.000397418328754611\\
0.55693972985354	0.00470515930666168	0.0303840192119186	-0.000359616957602831\\
0.587323749065458	0.00434554234905885	0.029737653719981	-0.00032615000831191\\
0.617061402785439	0.00401939234074694	0.0288895896225184	-0.000296444075893325\\
0.645950992407958	0.00372294826485361	0.0278645351864306	-0.000270048683277613\\
0.673815527594388	0.003452899581576	0.0266899093898374	-0.000246563053103832\\
0.700505436984226	0.00320633652847217	0.0253944384759119	-0.000225608602473976\\
0.725899875460138	0.00298072792599819	0.0240097288880299	-0.000206852483393945\\
0.749909604348168	0.00277387544260424	0.0225645349591661	-0.000190034862053493\\
0.772474139307334	0.00258384058055075	0.0210870075171489	-0.000174927192198631\\
0.793561146824483	0.00240891338835212	0.0116117726532851	-9.56368785302955e-05\\
0.805172919477768	0.00231327650982182	0.0111090372822511	-9.12750516722446e-05\\
0.816281956760019	0.00222200145814958	0.0106108564248715	-8.71610620797546e-05\\
0.82689281318489	0.00213484039606983	0.0101193806175669	-8.32777870462599e-05\\
};
\addlegendentry{data6}

\addplot[-Straight Barb, color=mycolor7, point meta={sqrt((\thisrow{u})^2+(\thisrow{v})^2)}, point meta min=0, quiver={u=\thisrow{u}, v=\thisrow{v}, every arrow/.append style={-{Straight Barb[angle'=18.263, scale={10/1000*\pgfplotspointmetatransformed}]}}}]
 table[row sep=crcr] {%
x	y	u	v\\
0.0429794912089172	0.0902926538286621	0.00200851347655998	-0.00441993812176299\\
0.0449880046854771	0.0858727157068992	0.00210786590931679	-0.00419550649207952\\
0.0470958705947939	0.0816772092148196	0.00221154875878551	-0.00398249274309212\\
0.0493074193535794	0.0776947164717275	0.00231970775421136	-0.00378034510077123\\
0.0516271271077908	0.0739143713709563	0.00599834404371817	-0.00826278130521162\\
0.057625471151509	0.0656515900657447	0.0067063626421305	-0.00729742877315877\\
0.0643318337936395	0.0583541612925859	0.00748423430275116	-0.00644455031650067\\
0.0718160680963906	0.0519096109760852	0.00833587693141746	-0.00569308090727606\\
0.0801519450278081	0.0462165300688092	0.00926454024089592	-0.0050310222376903\\
0.089416485268704	0.0411855078311189	0.0102723729195462	-0.00444543110385405\\
0.0996888581882503	0.0367400767272648	0.0113596122732091	-0.00392836099299552\\
0.111048470461459	0.0328117157342693	0.0125255572952834	-0.0034729126415804\\
0.123574027756743	0.0293388030926889	0.0137678937707937	-0.00307175559020102\\
0.137341921527536	0.0262670475024879	0.0150815138298986	-0.00271725858604265\\
0.152423435357435	0.0235497889164452	0.0164572462813613	-0.00240451563092741\\
0.168880681638796	0.0211452732855178	0.0178838835033951	-0.00212919821348538\\
0.186764565142191	0.0190160750720325	0.0193478496244678	-0.00188682030294255\\
0.206112414766659	0.0171292547690899	0.0208307839159879	-0.001672895959844\\
0.226943198682647	0.0154563588092459	0.0223100465129698	-0.00148437361709338\\
0.249253245195617	0.0139719851921525	0.0237607978742381	-0.00131852228187015\\
0.273014043069855	0.0126534629102824	0.0251566812248749	-0.00117258769961318\\
0.29817072429473	0.0114808752106692	0.0264669965830706	-0.00104392430338042\\
0.3246377208778	0.0104369509072888	0.0276619126469271	-0.000930621895690837\\
0.352299633524727	0.00950632901159794	0.028711818176404	-0.000830955911270113\\
0.381011451701131	0.00867537310032782	0.0295887524283685	-0.000743232142718918\\
0.4106002041295	0.0079321409576089	0.0302664367296624	-0.000665880345694855\\
0.440866640859162	0.00726626061191405	0.0307273088904237	-0.000597709111369832\\
0.471593949749586	0.00666855150054422	0.0309578020221387	-0.000537642346276563\\
0.502551751771725	0.00613090915426765	0.0309489309405339	-0.000484645287147212\\
0.533500682712259	0.00564626386712044	0.0307008521227414	-0.000437785795445881\\
0.564201534835	0.00520847807167456	0.0302231812638362	-0.000396336125500979\\
0.594424716098836	0.00481214194617358	0.0295302216153119	-0.000359643784551177\\
0.623954937714148	0.0044524981616224	0.0286400094965396	-0.000327089855956818\\
0.652594947210688	0.00412540830566559	0.027578931691034	-0.000298127129927256\\
0.680173878901722	0.00382728117573833	0.0263752087892629	-0.000272325782714267\\
0.706549087690985	0.00355495539302406	0.0250586222305076	-0.000249304793089123\\
0.731607709921492	0.00330565059993494	0.0236591257971249	-0.000228704363690854\\
0.755266835718617	0.00307694623624409	0.0222068317913232	-0.000210208217892795\\
0.77747366750994	0.00286673801835129	0.020727947244624	-0.000193570580391159\\
0.798201614754564	0.00267316743796013	0.01924749693478	-0.000178575893279454\\
0.817449111689344	0.00249459154468068	0.0105073789138159	-9.74115781716463e-05\\
0.82795649060316	0.00239717996650903	0.010021413865195	-9.30936562775707e-05\\
0.837977904468355	0.00230408631023146	0.00954401640916347	-8.90124559246898e-05\\
0.847521920877519	0.00221507385430677	0.00907681583305542	-8.51520642225693e-05\\
};
\addlegendentry{data7}

\addplot[-Straight Barb, color=mycolor1, point meta={sqrt((\thisrow{u})^2+(\thisrow{v})^2)}, point meta min=0, quiver={u=\thisrow{u}, v=\thisrow{v}, every arrow/.append style={-{Straight Barb[angle'=18.263, scale={10/1000*\pgfplotspointmetatransformed}]}}}]
 table[row sep=crcr] {%
x	y	u	v\\
0.0486196736100469	0.0873849377069785	0.00227766637356272	-0.00427709286257939\\
0.0508973399836096	0.0831078448443991	0.00238944143777269	-0.00405894601236269\\
0.0532867814213823	0.0790488988320364	0.00250596871852071	-0.00385195607173758\\
0.055792750139903	0.0751969427602988	0.00262739293865839	-0.00365558092335311\\
0.0584201430785614	0.0715413618369457	0.00674981118525662	-0.00794664589409387\\
0.065169954263818	0.0635947159428519	0.00753233776110553	-0.00701921473525305\\
0.0727022920249235	0.0565755012075988	0.00838851601953812	-0.00619990005245206\\
0.0810908080444617	0.0503756011551468	0.00932150928652911	-0.00547800659648361\\
0.0904123173309908	0.0448975945586631	0.0103335944650446	-0.00484199792781276\\
0.100745911796035	0.0400555966308504	0.0114255949875419	-0.00427954156111892\\
0.112171506783577	0.0357760550697315	0.0125958425115861	-0.00378297556094184\\
0.124767349295163	0.0319930795087896	0.013841436170272	-0.00334561900004238\\
0.138608785465435	0.0286474605087473	0.0151576014833146	-0.00296041949929043\\
0.15376638694875	0.0256870410094568	0.016536220734043	-0.00262011279017106\\
0.170302607682793	0.0230669282192858	0.0179647077265917	-0.00231995913140831\\
0.188267315409385	0.0207469690878775	0.0194282275707783	-0.00205575946870987\\
0.207695542980163	0.0186912096191676	0.0209095867386787	-0.00182319372048826\\
0.228605129718842	0.0168680158986793	0.0223865360873362	-0.00161798927643464\\
0.250991665806178	0.0152500266222447	0.0238333269563538	-0.00143719142025438\\
0.274824992762532	0.0138128352019903	0.025222370746791	-0.00127814788260294\\
0.300047363509323	0.0125346873193874	0.0265252181858502	-0.00113820336213419\\
0.326572581695173	0.0113964839572532	0.0277101257416742	-0.00101483301476759\\
0.354282707436847	0.0103816509424856	0.0287483867204479	-0.000906186220717212\\
0.383031094157295	0.00947546472176837	0.0296125441202014	-0.000810590583871977\\
0.412643638277496	0.00866487413789639	0.0302777702652792	-0.000726413531781399\\
0.442921408542776	0.00793846060611499	0.0307231807966535	-0.000652154497370211\\
0.473644589339429	0.00728630610874478	0.030937817008298	-0.000586663448227035\\
0.504582406347727	0.00669964266051775	0.0309153436326259	-0.000528903018697331\\
0.535497749980353	0.00617073964182042	0.030654205872881	-0.000477880125357287\\
0.566151955853234	0.00569285951646313	0.0301628242703457	-0.00043270551048816\\
0.59631478012358	0.00526015400597497	0.0294575005661689	-0.00039268230459007\\
0.625772280689749	0.0048674717013849	0.0285586189262748	-0.000357187441231821\\
0.654330899616023	0.00451028426015308	0.0274894483583175	-0.000325631147512642\\
0.681820347974341	0.00418465311264044	0.0262797695229777	-0.000297493446318732\\
0.708100117497319	0.0038871596663217	0.0249590860436903	-0.00027236649918862\\
0.733059203541009	0.00361479316713308	0.0235575234888646	-0.000249889875479729\\
0.756616727029874	0.00336490329165336	0.022104511992926	-0.000229723085095652\\
0.7787212390228	0.0031351802065577	0.0206275115180442	-0.000211566546457302\\
0.799348750540844	0.0029236136601004	0.0191497207725678	-0.000195188517014594\\
0.818498471313412	0.00272842514308581	0.017692879337006	-0.000180386171600841\\
0.836191350650418	0.00254803897148497	0.00958626425643694	-9.8152959326559e-05\\
0.845777614906855	0.00244988601215841	0.00912137358618037	-9.3904026979071e-05\\
0.854898988493035	0.00235598198517934	0.0086675795552329	-8.98808268303807e-05\\
0.863566568048268	0.00226610115834896	0.00822609611097047	-8.60685781814861e-05\\
};
\addlegendentry{data8}

\addplot[-Straight Barb, color=mycolor2, point meta={sqrt((\thisrow{u})^2+(\thisrow{v})^2)}, point meta min=0, quiver={u=\thisrow{u}, v=\thisrow{v}, every arrow/.append style={-{Straight Barb[angle'=18.263, scale={10/1000*\pgfplotspointmetatransformed}]}}}]
 table[row sep=crcr] {%
x	y	u	v\\
0.0540640817455598	0.0841253532831181	0.00253977798329158	-0.00411707890025806\\
0.0566038597288513	0.0800082743828601	0.00266342419585358	-0.00390619567173367\\
0.0592672839247049	0.0761020787111264	0.00279219604360476	-0.00370615371902322\\
0.0620594799683097	0.0723959249921032	0.00292623277109483	-0.00351642345800103\\
0.0649857127394045	0.0688795015341021	0.00747048547796669	-0.00760425062866828\\
0.0724561982173712	0.0612752509054339	0.00832133097064278	-0.00671772106391248\\
0.080777529188014	0.0545575298415214	0.00924849271800705	-0.00593460435942055\\
0.090026021906021	0.0486229254821008	0.0102542145068318	-0.00524461799959634\\
0.100280236412853	0.0433783074825045	0.0113396417584534	-0.00463672444106157\\
0.111619878171306	0.0387415830414429	0.0125041062343451	-0.00409922200962001\\
0.124123984405651	0.0346423610318229	0.0137439233397181	-0.00362476053313842\\
0.13786790774537	0.0310176004986845	0.0150539092155479	-0.00320690668932866\\
0.152921816960917	0.0278106938093558	0.0164268031475608	-0.00283890931207174\\
0.169348620108478	0.0249717844972841	0.0178515179283039	-0.00251387971584156\\
0.187200138036782	0.0224579047814425	0.0193123116274525	-0.00222726261220395\\
0.206512449664235	0.0202306421692386	0.0207910893828694	-0.00197500850772715\\
0.227303539047104	0.0182556336615114	0.0222675257416204	-0.00175297797769164\\
0.249571064788725	0.0165026556838198	0.0237162022600412	-0.00155711582544937\\
0.273287267048766	0.0149455398583704	0.0251092421244908	-0.00138457724649016\\
0.298396509173257	0.0135609626118803	0.0264174161374496	-0.00123280289303839\\
0.324813925310706	0.0123281597188419	0.0276113414399285	-0.00109924646587499\\
0.352425266750635	0.0112289132529669	0.0286596195254842	-0.000981507207618785\\
0.381084886276119	0.0102474060453481	0.0295358326000504	-0.000877803785583085\\
0.410620718876169	0.00936960225976501	0.0302157413280359	-0.000786525876590054\\
0.440836460204205	0.00858307638317496	0.0306785113042174	-0.000706109616551643\\
0.471514971508423	0.00787696676662331	0.0309091716363117	-0.000635127336939149\\
0.502424143144734	0.00724183942968416	0.0309031438301089	-0.000572476770333747\\
0.533327286974843	0.00666936265935042	0.0306607639501464	-0.000517165176923281\\
0.56398805092499	0.00615219748242714	0.0301870583599524	-0.000468245420858457\\
0.594175109284942	0.00568395206156868	0.0294971286231974	-0.000424873285637019\\
0.62367223790814	0.00525907877593166	0.0286121861479627	-0.000386385884705162\\
0.652284424056102	0.0048726928912265	0.0275568108560784	-0.000352192072339106\\
0.679841234912181	0.00452050081888739	0.0263576265840637	-0.00032173300333985\\
0.706198861496244	0.00419876781554754	0.0250458561255641	-0.000294516788013439\\
0.731244717621809	0.0039042510275341	0.0236509043127296	-0.000270158442291323\\
0.754895621934538	0.00363409258524278	0.0222020809195262	-0.000248318599336535\\
0.777097702854064	0.00338577398590625	0.0207273629144837	-0.000228676396940115\\
0.797825065768548	0.00315709758896613	0.0192511671612035	-0.000210949075926689\\
0.817076232929752	0.00294614851303944	0.0177937035398547	-0.000194918848731414\\
0.834869936469606	0.00275122966430803	0.016373633499005	-0.000180395388609787\\
0.851243569968611	0.00257083427569824	0.00881251922101156	-9.78991496165837e-05\\
0.860056089189623	0.00247293512608166	0.00836976105709897	-9.37452841727427e-05\\
0.868425850246722	0.00237918984190891	0.0079396656488705	-8.9806063927896e-05\\
0.876365515895592	0.00228938377798102	0.00752312195734195	-8.60677789698478e-05\\
};
\addlegendentry{data9}

\addplot[-Straight Barb, color=mycolor3, point meta={sqrt((\thisrow{u})^2+(\thisrow{v})^2)}, point meta min=0, quiver={u=\thisrow{u}, v=\thisrow{v}, every arrow/.append style={-{Straight Barb[angle'=18.263, scale={10/1000*\pgfplotspointmetatransformed}]}}}]
 table[row sep=crcr] {%
x	y	u	v\\
0.059290792905464	0.0805270257531059	0.00279398812470064	-0.00394054329330819\\
0.0620847810301647	0.0765864824597977	0.00292892823326682	-0.00373787811626336\\
0.0650137092634315	0.0728486043435343	0.00306932069452609	-0.00354568428236263\\
0.0680830299579576	0.0693029200611717	0.00321529527844737	-0.00336344764106328\\
0.071298325236405	0.0659394724201084	0.00815922093081906	-0.0072371932471902\\
0.079457546167224	0.0587022791729182	0.00907245469181787	-0.00639433730283567\\
0.0885300008590419	0.0523079418700825	0.0100636754198394	-0.00564986686450493\\
0.0985936762788813	0.0466580750055776	0.0111340861604564	-0.00499395252335378\\
0.109727762439338	0.0416641224822238	0.0122835917843089	-0.00441609008223846\\
0.122011354223647	0.0372480323999854	0.0135099257405955	-0.00390522731995412\\
0.135521279964242	0.0333428050800312	0.0148073506105671	-0.00345435151794163\\
0.150328630574809	0.0298884535620896	0.0161683976517064	-0.00305730434491146\\
0.166497028226516	0.0268311492171782	0.0175833754935238	-0.00270765745607732\\
0.184080403720039	0.0241234917611008	0.019038361220291	-0.00239890647478426\\
0.20311876494033	0.0217245852863166	0.0205147880786205	-0.00212669792907341\\
0.223633553018951	0.0195978873572432	0.0219917292563456	-0.00188715064567749\\
0.245625282275296	0.0177107367115657	0.0234462471160884	-0.00167631977662107\\
0.269071529391385	0.0160344169349446	0.0248504779693959	-0.00149037247550951\\
0.293922007360781	0.0145440444594351	0.0261752903780914	-0.00132658628412523\\
0.320097297738872	0.0132174581753099	0.027390761512571	-0.00118250786507455\\
0.347488059251443	0.0120349503102353	0.0284675191114334	-0.00105570943629963\\
0.375955578362877	0.0109792408739357	0.0293755588685151	-0.000943918367514443\\
0.405331137231392	0.0100353225064212	0.0300914900725482	-0.000845431480299381\\
0.43542262730394	0.00918989102612186	0.0305948877936872	-0.000758709251602136\\
0.466017515097627	0.00843118177451972	0.0308693064982549	-0.000682262928387354\\
0.496886821595882	0.00774891884613237	0.0309056865851459	-0.000614741036298258\\
0.527792508181028	0.00713417780983411	0.0307052741525846	-0.000555094468642506\\
0.558497782333613	0.00657908334119161	0.0302742679505504	-0.000502379918406687\\
0.588772050284163	0.00607670342278492	0.0296232794445139	-0.000455699755638966\\
0.618395329728677	0.00562100366714595	0.028772587530239	-0.000414256732531932\\
0.647167917258916	0.00520674693461402	0.0277468474242126	-0.000377424329254856\\
0.674914764683129	0.00482932260535916	0.0265733633552212	-0.000344645151649352\\
0.70148812803835	0.00448467745370981	0.0252807135854498	-0.000315392632157034\\
0.7267688416238	0.00416928482155278	0.0239002736881615	-0.000289203683383402\\
0.750669115311961	0.00388008113816938	0.0224605391077688	-0.000265716759595793\\
0.77312965441973	0.00361436437857358	0.0209893868421057	-0.000244613828827551\\
0.794119041261836	0.00336975054974603	0.0195129047559495	-0.000225593870183297\\
0.813631946017785	0.00314415667956274	0.018052467062512	-0.000208391088842909\\
0.831684413080297	0.00293576559071983	0.0166255208262677	-0.000192801573612189\\
0.848309933906565	0.00274296401710764	0.0152479861462447	-0.00017864731021844\\
0.863557920052809	0.0025643167068892	0.00815770569195795	-9.66893789510694e-05\\
0.871715625744767	0.00246762732793813	0.00773655876086865	-9.26567606716433e-05\\
0.879452184505636	0.00237497056726648	0.00732901185881885	-8.88274642330488e-05\\
0.886781196364455	0.00228614310303344	0.00693570102210517	-8.51888017389065e-05\\
};
\addlegendentry{data10}

\addplot[-Straight Barb, color=mycolor4, point meta={sqrt((\thisrow{u})^2+(\thisrow{v})^2)}, point meta min=0, quiver={u=\thisrow{u}, v=\thisrow{v}, every arrow/.append style={-{Straight Barb[angle'=18.263, scale={10/1000*\pgfplotspointmetatransformed}]}}}]
 table[row sep=crcr] {%
x	y	u	v\\
0.0642787609686539	0.0766044443118978	0.00303943933364476	-0.00374819891159778\\
0.0673182003022987	0.0728562454003	0.00318507447491552	-0.00355467662475256\\
0.0705032747772142	0.0693015687755475	0.00333644415552854	-0.00337120216910704\\
0.0738397189327427	0.0659303666064404	0.00349366519132432	-0.00319727973217811\\
0.0773333841240671	0.0627330868742623	0.00881496709358397	-0.00684708442422201\\
0.086148351217651	0.0558860024500403	0.00978494234924633	-0.00605046848093752\\
0.0959332935668974	0.0498355339691028	0.0108337185404137	-0.00534690894987104\\
0.106767012107311	0.0444886250192318	0.0119613757158138	-0.00472706737734225\\
0.118728387823125	0.0397615576418895	0.0131665120236693	-0.00418100493815338\\
0.131894899846794	0.0355805527037361	0.0144452071253148	-0.00369833619741212\\
0.146340106972109	0.031882216506324	0.0157896922038298	-0.00327241010950219\\
0.162129799175939	0.0286098063968218	0.0171902732113326	-0.00289736892361355\\
0.179320072387271	0.0257124374732083	0.018634941881792	-0.00256712729672284\\
0.197955014269063	0.0231453101764854	0.0201071363543694	-0.00227557389514746\\
0.218062150623433	0.020869736281338	0.021585828935684	-0.00201857354974341\\
0.239647979559117	0.0188511627315946	0.0230477069682934	-0.00179243088421949\\
0.26269568652741	0.0170587318473751	0.0244677327399721	-0.00159340875348785\\
0.287163419267382	0.0154653230938872	0.0258162768034037	-0.00141790211148798\\
0.312979696070786	0.0140474209823992	0.0270636918808936	-0.00126332255111132\\
0.34004338795168	0.0127840984312879	0.0281801302748692	-0.0011273347430943\\
0.368223518226549	0.0116567636881936	0.0291369612163093	-0.00100763832232723\\
0.397360479442858	0.0106491253658664	0.0299063150547014	-0.000902093118206417\\
0.427266794497559	0.00974703224765998	0.0304682340447918	-0.000809082098978125\\
0.457735028542351	0.00893795014868186	0.0308063609654634	-0.000727144178490492\\
0.488541389507815	0.00821080597019136	0.0309087123261591	-0.000654871429933975\\
0.519450101833974	0.00755593454025739	0.0307718216674436	-0.000590991738164586\\
0.550221923501417	0.0069649428020928	0.0304020617666868	-0.000534513687726649\\
0.580623985268104	0.00643042911436615	0.0298106198670474	-0.000484547318514404\\
0.610434605135151	0.00594588179585175	0.0290127084288615	-0.000440247459198051\\
0.639447313564013	0.0055056343366537	0.0280324783850714	-0.000400865513709611\\
0.667479791949084	0.00510476882294409	0.0268968790536191	-0.000365813170023452\\
0.694376671002704	0.00473895565292064	0.025634838273757	-0.000334568141150083\\
0.720011509276461	0.00440438751177055	0.0242758836676311	-0.000306637171981932\\
0.744287392944092	0.00409775033978862	0.0228507312750311	-0.00028158658503979\\
0.767138124219123	0.00381616375474883	0.0213865276392359	-0.000259078676209546\\
0.788524651858359	0.00355708507853928	0.0199094331150335	-0.000238816905171238\\
0.808434084973392	0.00331826817336805	0.018443498639636	-0.000220520247754106\\
0.826877583613028	0.00309774792561394	0.0170072431627377	-0.000203940030096024\\
0.843884826775766	0.00289380789551792	0.0156156502950998	-0.00018888615725343\\
0.859500477070866	0.00270492173826449	0.0142822597794369	-0.000175192793301937\\
0.873782736850302	0.00252972894496255	0.00759974179843859	-9.45646382226667e-05\\
0.881382478648741	0.00243516430673988	0.00719893715525066	-9.06789780320671e-05\\
0.888581415803992	0.00234448532870782	0.00681225724736867	-8.6984997409942e-05\\
0.89539367305136	0.00225750033129787	0.00644014120810599	-8.34709913481815e-05\\
};
\addlegendentry{data11}

\addplot[-Straight Barb, color=mycolor5, point meta={sqrt((\thisrow{u})^2+(\thisrow{v})^2)}, point meta min=0, quiver={u=\thisrow{u}, v=\thisrow{v}, every arrow/.append style={-{Straight Barb[angle'=18.263, scale={10/1000*\pgfplotspointmetatransformed}]}}}]
 table[row sep=crcr] {%
x	y	u	v\\
0.0690079011482112	0.072373403810507	0.00327527732051837	-0.00354082155617723\\
0.0722831784687296	0.0688325822543298	0.00343099097292082	-0.00335733233810048\\
0.0757141694416504	0.0654752499162293	0.00359267943096213	-0.00318341494615208\\
0.0793068488726125	0.0622918349700772	0.00376044362531448	-0.00301859479333027\\
0.083067292497927	0.059273240176747	0.00943675120747216	-0.00643554205706542\\
0.0925040437053991	0.0528376981196815	0.0104581237264498	-0.00568753040398717\\
0.102962167431849	0.0471501677156944	0.0115583870766918	-0.00502696596205944\\
0.114520554508541	0.0421232017536349	0.0127364502767744	-0.0044450366179806\\
0.127257004785315	0.0376781651356543	0.0139895751742918	-0.00393239865961341\\
0.141246579959607	0.0337457664760409	0.0153121751503241	-0.003479349583453\\
0.156558755109931	0.0302664168925879	0.016694515508116	-0.00307962278233313\\
0.173253270618047	0.0271867941102548	0.0181247811761565	-0.00272768494615428\\
0.191378051794204	0.0244591091641005	0.0195887998916255	-0.00241781255659764\\
0.210966851685829	0.0220412966075029	0.0210676059358998	-0.00214429595841013\\
0.232034457621729	0.0198970006490927	0.0225380797064144	-0.00190323384732582\\
0.254572537328143	0.0179937668017669	0.0239749575999161	-0.00169113287350014\\
0.27854749492806	0.0163026339282668	0.0253515797865994	-0.00150447590834405\\
0.303899074714659	0.0147981580199227	0.0266371571800311	-0.00133989140406233\\
0.33053623189469	0.0134582666158604	0.027802124452421	-0.00119493590621406\\
0.358338356347111	0.0122633307096464	0.028817304387851	-0.00106740301913405\\
0.387155660734962	0.0111959276905123	0.029655344336155	-0.000955128024670411\\
0.416811005071117	0.0102407996658419	0.0302909882664111	-0.000856107577875583\\
0.447101993337528	0.00938469208796631	0.0307078705902489	-0.000768817920196297\\
0.477809863927777	0.00861587416777001	0.0308937148928201	-0.000691883041655973\\
0.508703578820597	0.00792399112611404	0.0308408613918969	-0.000623980886261639\\
0.539544440212494	0.0073000102398524	0.0305509473378948	-0.000563921627456091\\
0.570095387550389	0.00673608861239631	0.030034744987617	-0.000510775327506776\\
0.600130132538006	0.00622531328488953	0.0293075892217373	-0.000463708555794409\\
0.629437721759743	0.00576160472909512	0.0283883982625747	-0.000421931040919487\\
0.657826120022318	0.00533967368817564	0.0273041194159525	-0.000384744290653012\\
0.68513023943827	0.00495492939752263	0.0260831243947475	-0.00035159962922645\\
0.711213363833018	0.00460332976829618	0.0247551666946665	-0.000322010885854374\\
0.735968530527684	0.0042813188824418	0.0233500247569404	-0.00029551891930251\\
0.759318555284624	0.00398579996313929	0.0218972736314005	-0.00027171998457334\\
0.781215828916025	0.00371407997856595	0.0204225071090016	-0.000250300504403041\\
0.801638336025027	0.00346377947416291	0.0189500846370632	-0.000230985521577918\\
0.82058842066209	0.00323279395258499	0.0175020357054261	-0.000213514131487983\\
0.838090456367516	0.00301927982109701	0.0160942896810522	-0.00019765494814642\\
0.854184746048568	0.00282162487295059	0.014739676431939	-0.000183231656153835\\
0.868924422480507	0.00263839321679676	0.0134496960765758	-0.000170090510035947\\
0.882374118557083	0.00246830270676081	0.00712151958046292	-9.15681832129034e-05\\
0.889495638137546	0.00237673452354791	0.00673946549648075	-8.78543329089392e-05\\
0.896235103634026	0.00228888019063897	0.0063717902543805	-8.43201563043893e-05\\
0.902606893888407	0.00220456003433458	0.00601877925088079	-8.09548994819931e-05\\
};
\addlegendentry{data12}

\addplot[-Straight Barb, color=mycolor6, point meta={sqrt((\thisrow{u})^2+(\thisrow{v})^2)}, point meta min=0, quiver={u=\thisrow{u}, v=\thisrow{v}, every arrow/.append style={-{Straight Barb[angle'=18.263, scale={10/1000*\pgfplotspointmetatransformed}]}}}]
 table[row sep=crcr] {%
x	y	u	v\\
0.0734591708657533	0.0678509411557132	0.0035006516306717	-0.00331924683058384\\
0.076959822496425	0.0645316943251294	0.00366581327035717	-0.00314664126954025\\
0.0806256357667822	0.0613850530555891	0.00383715105479723	-0.00298308049711758\\
0.0844627868215794	0.0584019725584716	0.00401474766927028	-0.00282811399956425\\
0.0884775344908497	0.0555738585589073	0.0100236603879846	-0.00600418645639084\\
0.0985011948788343	0.0495696721025165	0.0110914005172669	-0.00530694561491764\\
0.109592595396101	0.0442627264875988	0.0122375244310403	-0.00469128388172428\\
0.121830119827141	0.0395714426058745	0.0134597517807505	-0.00414894846629228\\
0.135289871607892	0.0354224941395823	0.0147540093165364	-0.00367121834965873\\
0.150043880924428	0.0317512757899235	0.0161130670453454	-0.00324908901534528\\
0.166156947969774	0.0285021867745783	0.0175253221633773	-0.00287669961666482\\
0.183682270133151	0.0256254871579134	0.0189769736096742	-0.00254886346758823\\
0.202659243742825	0.0230766236903252	0.0204518619163171	-0.00226023639112957\\
0.223111105659142	0.0208163872991956	0.0219288714615051	-0.00200551885177034\\
0.245039977120647	0.0188108684474253	0.0233831456068562	-0.00178105800489477\\
0.268423122727504	0.0170298104425305	0.0247878743527949	-0.00158357753140872\\
0.293210997080298	0.0154462329111218	0.0261152039166527	-0.00140979179943404\\
0.319326200996951	0.0140364411116878	0.0273337040739	-0.0012565684989927\\
0.346659905070851	0.0127798726126951	0.0284143543008427	-0.0011216191808429\\
0.375074259371694	0.0116582534318522	0.0293290856715935	-0.00100287626257082\\
0.404403345043287	0.0106553771692814	0.0300521923870207	-0.000898318203559595\\
0.434455537430308	0.00975705896572176	0.0305612999827663	-0.000806082712863063\\
0.465016837413074	0.0089509762528587	0.0308436187398322	-0.000724745600486477\\
0.495860456152907	0.00822623065237222	0.0308908065361723	-0.00065302166865888\\
0.526751262689079	0.00757320898371334	0.0306992594047391	-0.000589679142384227\\
0.557450522093818	0.00698352984132911	0.0302751541745158	-0.000533613129396183\\
0.587725676268334	0.00644991671193293	0.0296329689364583	-0.000483958320381941\\
0.617358645204792	0.00596595839155099	0.0287914267585028	-0.000439940395544889\\
0.646150071963295	0.0055260179960061	0.0277723757493128	-0.000400825994327412\\
0.673922447712607	0.00512519200167869	0.0266047036289653	-0.000365967970876723\\
0.700527151341573	0.00475922403080196	0.0253175612995237	-0.000334858395842771\\
0.725844712641096	0.00442436563495919	0.023940965893971	-0.000307047966380476\\
0.749785678535067	0.00411731766857872	0.0225044776988711	-0.000282112279702117\\
0.772290156233938	0.0038352053888766	0.0210362735472909	-0.000259677975321015\\
0.793326429781229	0.00357552741355558	0.0195603374055306	-0.000239455808083885\\
0.81288676718676	0.0033360716054717	0.0180992599223823	-0.000221192457238925\\
0.830986027109142	0.00311487914823277	0.0166731532436748	-0.000204647240671063\\
0.847659180352817	0.00291023190756171	0.0152956446488455	-0.000189606231775671\\
0.862954825001663	0.00272062567578604	0.0139777021943801	-0.000175906879862458\\
0.876932527196043	0.00254471879592358	0.0127290906482815	-0.000163407479679168\\
0.889661617844324	0.00238131131624441	0.00670972618807719	-8.77458489404979e-05\\
0.896371344032401	0.00229356546730392	0.00634472700193112	-8.42275236656405e-05\\
0.902716071034332	0.00220933794363827	0.00599418470414614	-8.08764839913471e-05\\
0.908710255738479	0.00212846145964693	0.00565826417084825	-7.76829008850219e-05\\
};
\addlegendentry{data13}

\addplot[-Straight Barb, color=mycolor7, point meta={sqrt((\thisrow{u})^2+(\thisrow{v})^2)}, point meta min=0, quiver={u=\thisrow{u}, v=\thisrow{v}, every arrow/.append style={-{Straight Barb[angle'=18.263, scale={10/1000*\pgfplotspointmetatransformed}]}}}]
 table[row sep=crcr] {%
x	y	u	v\\
0.0776146464291757	0.0630552667084523	0.00371471657208038	-0.00308436677577792\\
0.0813293630012561	0.0599708999326743	0.00388868494682598	-0.00292345110627197\\
0.085218047948082	0.0570474488264024	0.00406899518994684	-0.00277100399576098\\
0.0892870431380289	0.0542764448306414	0.00425570996483614	-0.00262660178794748\\
0.093542753102865	0.0516498430426939	0.0105748241363689	-0.00555463637562434\\
0.104117577239234	0.0460952066670696	0.0116842224090392	-0.00491013996825152\\
0.115801799648273	0.041185066698818	0.0128710212209007	-0.00434111640923945\\
0.128672820869174	0.0368439502895786	0.0141317579747127	-0.00383990286506875\\
0.142804578843887	0.0330040474245098	0.0154610510485268	-0.00339842653468871\\
0.158265629892413	0.0296056208898211	0.0168500785136249	-0.00300839493286171\\
0.175115708406038	0.0265972259569594	0.018285489602667	-0.00266437285655446\\
0.193401198008705	0.023932853100405	0.0197516516796547	-0.00236154079066628\\
0.21315284968836	0.0215713123097387	0.0212306068084165	-0.00209495014674042\\
0.234383456496776	0.0194763621629983	0.0226993473786103	-0.00185971967137642\\
0.257082803875387	0.0176166424916218	0.0241316077098141	-0.00165245857111173\\
0.281214411585201	0.0159641839205101	0.0254993940227963	-0.00147012135408847\\
0.306713805607997	0.0144940625664216	0.0267740289980291	-0.00130966451458956\\
0.333487834606026	0.0131843980518321	0.0279238677497857	-0.00116820060368676\\
0.361411702355812	0.0120161974481453	0.0289207749846485	-0.00104360535878515\\
0.39033247734046	0.0109725920893602	0.0297380909217406	-0.000933958879945478\\
0.420070568262201	0.0100386332094147	0.0303519853985691	-0.000837389626143001\\
0.45042255366077	0.00920124358327169	0.0307430694559964	-0.000752180341542664\\
0.481165623116766	0.00844906324172903	0.0309019935376048	-0.000677011918397379\\
0.512067616654371	0.00777205132333165	0.030824103771817	-0.000610694895507336\\
0.542891720426188	0.00716135642782431	0.0305095127081551	-0.000552091600624806\\
0.573401233134343	0.00660926482719951	0.0299683592472099	-0.000500184422306698\\
0.603369592381553	0.00610908040489281	0.0292181987238741	-0.000454175288369626\\
0.632587791105427	0.00565490511652318	0.0282804829218866	-0.000413351075067345\\
0.660868274027314	0.00524155404145584	0.0271793387400148	-0.00037703698120612\\
0.688047612767329	0.00486451706024972	0.0259449314720203	-0.000344638262577917\\
0.713992544239349	0.0045198787976718	0.0246067290798556	-0.000315688597265279\\
0.738599273319204	0.00420419020040652	0.023194629398236	-0.000289776107671579\\
0.761793902717441	0.00391441409273494	0.0217376740618106	-0.000266511616727195\\
0.783531576779251	0.00364790247600775	0.0202625338097515	-0.000245552539307334\\
0.803794110589003	0.00340234993670041	0.0187916150582641	-0.000226634111076558\\
0.822585725647267	0.00317571582562385	0.0173458368809918	-0.000209524677841298\\
0.839931562528259	0.00296619114778256	0.0159435426308935	-0.000194003868785725\\
0.855875105159152	0.00277218727899683	0.0145963385935165	-0.000179875387000539\\
0.870471443752668	0.00259231189199629	0.0133135942748681	-0.000166990449302309\\
0.883785038027537	0.00242532144269398	0.0121036035562292	-0.000155219369158466\\
0.895888641583766	0.00227010207353552	0.00635393229424375	-8.31462042983461e-05\\
0.90224257387801	0.00218695586923717	0.00600429973927763	-7.98457784149877e-05\\
0.908246873617287	0.00210711009082218	0.00566908986726256	-7.66998673266059e-05\\
0.91391596348455	0.00203041022349558	0.00534837265079036	-7.3699544979702e-05\\
};
\addlegendentry{data14}

\addplot[-Straight Barb, color=mycolor1, point meta={sqrt((\thisrow{u})^2+(\thisrow{v})^2)}, point meta min=0, quiver={u=\thisrow{u}, v=\thisrow{v}, every arrow/.append style={-{Straight Barb[angle'=18.263, scale={10/1000*\pgfplotspointmetatransformed}]}}}]
 table[row sep=crcr] {%
x	y	u	v\\
0.0814575952050336	0.0580056909571198	0.00391663247894511	-0.00283712628194558\\
0.0853742276839787	0.0551685646751742	0.00409875842299279	-0.00268865781424171\\
0.0894729861069715	0.0524799068609325	0.00428735990388286	-0.00254803470562794\\
0.0937603460108543	0.0499318721553046	0.00448247837417409	-0.00241486285283891\\
0.0982428243850284	0.0475170093024657	0.0110893974813923	-0.0050885058119609\\
0.109332221866421	0.0424285034905048	0.0122360640285054	-0.00449853976471678\\
0.121568285894926	0.037929963725788	0.0134587854045307	-0.00397772242415235\\
0.135027071299457	0.0339522413016357	0.0147529454464535	-0.00351900924497323\\
0.14978001674591	0.0304332320566624	0.0161119017378525	-0.00311499919983709\\
0.165891918483763	0.0273182328568253	0.0175253143132215	-0.00275812492565122\\
0.183417232796984	0.0245601079311741	0.0189782188821817	-0.00244339530673733\\
0.202395451679166	0.0221167126244368	0.0204533159756848	-0.00216637693975753\\
0.222848767654851	0.0199503356846792	0.0219310407496666	-0.00192253182854525\\
0.244779808404517	0.018027803856134	0.0233867444655473	-0.00170740481051328\\
0.268166552870065	0.0163203990456207	0.0247930457936885	-0.00151787970371984\\
0.292959598663753	0.0148025193419009	0.026121081797255	-0.00135115444908318\\
0.319080680461008	0.0134513648928177	0.027341661626578	-0.00120443743423518\\
0.346422342087586	0.0122469274585825	0.0284232621462812	-0.00107509145922902\\
0.374845604233867	0.0111718359993535	0.0293388665438423	-0.00096116475537091\\
0.40418447077771	0.0102106712439826	0.0300634096300922	-0.000860893006957955\\
0.434247880407802	0.00934977823702463	0.030575052153276	-0.000772560792817383\\
0.464822932561078	0.00857721744420725	0.030857371216894	-0.000694599559306843\\
0.495680303777972	0.0078826178849004	0.0309042447285985	-0.000625800069845913\\
0.526584548506571	0.00725681781505449	0.0307144046830896	-0.000565072798894846\\
0.55729895318966	0.00669174501615964	0.0302913092373707	-0.000511377460126881\\
0.587590262427031	0.00618036755603276	0.02964850517378	-0.000463785772985664\\
0.617238767600811	0.0057165817830471	0.0288060741551426	-0.000421569049242853\\
0.646044841755953	0.00529501273380425	0.0277876394249712	-0.000384077035402298\\
0.673832481180925	0.00491093569840195	0.0266190731804261	-0.000350694832254102\\
0.700451554361351	0.00456024086614785	0.0253313175539803	-0.000320880985831006\\
0.725782871915331	0.00423935988031684	0.023953835868589	-0.000294211444858696\\
0.74973670778392	0.00394514843545814	0.022516157485912	-0.000270312164832552\\
0.772252865269832	0.00367483627062559	0.0210466269407905	-0.000248829572445773\\
0.793299492210623	0.00342600669817982	0.0195703988625714	-0.00022945219369052\\
0.812869891073194	0.0031965545044893	0.0181083870127792	-0.000211939851717882\\
0.830978278085973	0.00298461465277142	0.0166799700053446	-0.000196082573496761\\
0.847658248091318	0.00278853207927466	0.0153018896511166	-0.000181680376260668\\
0.862960137742434	0.00260685170301399	0.0139839929496474	-0.00016855475662346\\
0.876944130692082	0.00243829694639053	0.0127342829305557	-0.000156570714217408\\
0.889678413622637	0.00228172623217312	0.0115598094341655	-0.00014561058015209\\
0.901238223056803	0.00213611565202103	0.0060459100995186	-7.78205864635432e-05\\
0.907284133156321	0.00205829506555749	0.0057100172859792	-7.47589399691692e-05\\
0.912994150442301	0.00198353612558832	0.00538844428410457	-7.18386670240246e-05\\
0.918382594726405	0.00191169745856429	0.00508118599904206	-6.90517259973934e-05\\
};
\addlegendentry{data15}

\addplot[-Straight Barb, color=mycolor2, point meta={sqrt((\thisrow{u})^2+(\thisrow{v})^2)}, point meta min=0, quiver={u=\thisrow{u}, v=\thisrow{v}, every arrow/.append style={-{Straight Barb[angle'=18.263, scale={10/1000*\pgfplotspointmetatransformed}]}}}]
 table[row sep=crcr] {%
x	y	u	v\\
0.0849725429949514	0.0527225467610502	0.00410556738150578	-0.00257851929145315\\
0.0890781103764572	0.0501440274695971	0.00429519610549932	-0.0024432020590471\\
0.0933733064819565	0.04770082541055	0.00449140571400865	-0.00231506261734836\\
0.0978647121959652	0.0453857627932016	0.00469421584248245	-0.00219373899709452\\
0.102558928038448	0.0431920237961071	0.0115665450549853	-0.00460740151196885\\
0.114125473093433	0.0385846222841383	0.0127464030743335	-0.00407356939395538\\
0.126871876167766	0.0345110528901829	0.0140007140336184	-0.00360236377966363\\
0.140872590201385	0.0309086891105192	0.015323754954978	-0.00318738447324143\\
0.156196345156363	0.0277213046372778	0.0167076869919682	-0.00282192387442049\\
0.172904032148331	0.0248993807628573	0.0181407432615771	-0.00249915191909459\\
0.191044775409908	0.0224002288437627	0.0196064882618182	-0.00221453857751157\\
0.210651263671726	0.0201856902662512	0.0210861237613218	-0.00196405391490064\\
0.231737387433048	0.0182216363513505	0.0225586656656218	-0.0017435843050648\\
0.25429605309867	0.0164780520462857	0.0239980601658261	-0.00154910806553099\\
0.278294113264496	0.0149289439807547	0.0253760650828915	-0.00137779513509676\\
0.303670178347388	0.013551148845658	0.0266632128227707	-0.00122709832031574\\
0.330333391170158	0.0123240505253422	0.0278300494750582	-0.00109448680187412\\
0.358163440645216	0.0112295637234681	0.0288454258058261	-0.000977578693514865\\
0.387008866451042	0.0102519850299533	0.0296835853013376	-0.000874602164661009\\
0.41669245175238	0.00937738286529224	0.0303211478127364	-0.000783955466409287\\
0.447013599565116	0.00859342739888296	0.0307382902049958	-0.000704084734007796\\
0.477751889770112	0.00788934266487516	0.0309214462449314	-0.000633573431884244\\
0.508673336015044	0.00725576923299092	0.0308674626801192	-0.000571326318137804\\
0.539540798695163	0.00668444291485311	0.0305781286316057	-0.000516357356516307\\
0.570118927326768	0.00616808555833681	0.0300598715086464	-0.000467726441277334\\
0.600178798835415	0.00570035911705947	0.0293291351795418	-0.000424596388862891\\
0.629507934014957	0.00527576272819658	0.0284080560945378	-0.000386309647880054\\
0.657915990109495	0.00488945308031653	0.0273219676997232	-0.000352280167439207\\
0.685237957809218	0.00453717291287732	0.0260980588783205	-0.000321954033813298\\
0.711336016687538	0.00421521887906402	0.024767680224935	-0.000294843823353337\\
0.736103696912473	0.00392037505571069	0.0233600761782141	-0.000270568259522619\\
0.759463773090687	0.00364980679618807	0.0219042573248249	-0.000248791421958417\\
0.781368030415512	0.00340101537422965	0.0204277802715411	-0.00022919563022542\\
0.801795810687053	0.00317181974400423	0.0189543364640682	-0.000211500806162454\\
0.820750147151122	0.00296031893784177	0.0175034612805848	-0.000195491434089244\\
0.838253608431706	0.00276482750375253	0.0160931391505579	-0.000180979224988515\\
0.854346747582264	0.00258384827876402	0.0147386816520338	-0.000167784649000892\\
0.869085429234298	0.00241606362976312	0.0134484139627521	-0.000155747149000976\\
0.88253384319705	0.00226031648076215	0.012229174905274	-0.000144745530125358\\
0.894763018102324	0.00211557095063679	0.0110869637991163	-0.000134674153177872\\
0.905849981901441	0.00198089679745892	0.00577912813628001	-7.18230497247995e-05\\
0.911629110037721	0.00190907374773412	0.00545543254630965	-6.90194509197588e-05\\
0.91708454258403	0.00184005429681436	0.00514591531588093	-6.63437351962263e-05\\
0.922230457899911	0.00177371056161813	0.0048505117178006	-6.37887294417478e-05\\
};
\addlegendentry{data16}

\addplot[-Straight Barb, color=mycolor3, point meta={sqrt((\thisrow{u})^2+(\thisrow{v})^2)}, point meta min=0, quiver={u=\thisrow{u}, v=\thisrow{v}, every arrow/.append style={-{Straight Barb[angle'=18.263, scale={10/1000*\pgfplotspointmetatransformed}]}}}]
 table[row sep=crcr] {%
x	y	u	v\\
0.0881453363447582	0.0472271074772683	0.00428069915101718	-0.0023095848082427\\
0.0924260354957754	0.0449175226690256	0.00447717195481727	-0.00218806545716716\\
0.0969032074505927	0.0427294572118584	0.00468030649989654	-0.00207301493641612\\
0.101583513950489	0.0406564422754423	0.00489010056799312	-0.00196410585048322\\
0.106473614518482	0.0386923364249591	0.0120054264036156	-0.00411292111625728\\
0.118479040922098	0.0345794153087018	0.0132146999513592	-0.00363664943596611\\
0.131693740873457	0.0309427658727357	0.0144966669250878	-0.00321630339696188\\
0.146190407798545	0.0277264624757738	0.0158445592755291	-0.00284615096279095\\
0.162034967074074	0.0248803115129829	0.0172494194145321	-0.00252019775746908\\
0.179284386488606	0.0223601137555138	0.0186981574798107	-0.00223236230012959\\
0.197982543968417	0.0201277514553842	0.0201730119923408	-0.00197859118965767\\
0.218155555960758	0.0181491602657265	0.0216538521533523	-0.00175527374695264\\
0.23980940811411	0.0163938865187739	0.0231184536561788	-0.00155873327746458\\
0.262927861770289	0.0148351532413093	0.024539573822606	-0.00138538849095105\\
0.287467435592895	0.0134497647503582	0.0258883245640806	-0.00123270589175931\\
0.313355760156975	0.0122170588585989	0.0271348486045566	-0.0010984034322148\\
0.340490608761532	0.0111186554263841	0.0282496230276626	-0.000980219118628693\\
0.368740231789195	0.0101384363077554	0.0292020482720227	-0.00087603105636523\\
0.397942280061217	0.0092624052513902	0.0299676934407481	-0.000784253955306251\\
0.427909973501965	0.00847815129608395	0.0305248770997898	-0.000703454747807562\\
0.458434850601755	0.00777469654827639	0.0308557472776939	-0.000632245904081924\\
0.489290597879449	0.00714245064419447	0.0309494215314385	-0.000569365854312777\\
0.520240019410888	0.00657308478988169	0.0308054329076347	-0.000513836868005308\\
0.551045452318522	0.00605924792187638	0.0304282925692231	-0.00046477937897204\\
0.581473744887745	0.00559446854290434	0.0298270316862356	-0.000421355816315593\\
0.611300776573981	0.00517311272658875	0.0290205330127064	-0.000382821593014416\\
0.640321309586687	0.00479029113357433	0.028032591362145	-0.000348591710919045\\
0.668353900948832	0.00444169942265529	0.0268898750469787	-0.000318145368835569\\
0.695243775995811	0.00412355405381972	0.025620551468732	-0.000290990501688427\\
0.720864327464543	0.00383256355213129	0.0242561174869614	-0.000266694323426724\\
0.745120444951504	0.00356586922870457	0.022825465925623	-0.000244918706838369\\
0.767945910877127	0.0033209505218662	0.0213570108665582	-0.000225366054101015\\
0.789302921743686	0.00309558446776518	0.0198774926338788	-0.000207754792801545\\
0.809180414377564	0.00288782967496364	0.018409231948204	-0.000191836477209648\\
0.827589646325768	0.00269599319775399	0.0169705149475298	-0.000177420299203773\\
0.844560161273298	0.00251857289855022	0.0155780823524135	-0.000164339646380261\\
0.860138243625712	0.00235423325216996	0.0142459829231187	-0.000152435494007065\\
0.87438422654883	0.00220179775816289	0.0129812326234665	-0.000141565368987371\\
0.887365459172297	0.00206023238917552	0.0117896815620627	-0.000131621907299419\\
0.89915514073436	0.0019286104818761	0.0106764435342953	-0.000122511518406343\\
0.909831584268655	0.00180609896346976	0.00554837715858314	-6.52102558618853e-05\\
0.915379961427238	0.00174088870760787	0.00523542683845823	-6.26822707933709e-05\\
0.920615388265696	0.0016782064368145	0.00493649579859701	-6.0268356589526e-05\\
0.925551884064293	0.00161793808022498	0.00465147175519587	-5.79621948952134e-05\\
};
\addlegendentry{data17}

\addplot[-Straight Barb, color=mycolor4, point meta={sqrt((\thisrow{u})^2+(\thisrow{v})^2)}, point meta min=0, quiver={u=\thisrow{u}, v=\thisrow{v}, every arrow/.append style={-{Straight Barb[angle'=18.263, scale={10/1000*\pgfplotspointmetatransformed}]}}}]
 table[row sep=crcr] {%
x	y	u	v\\
0.0909631995354518	0.0415415013001886	0.00444121818430489	-0.00203140272990386\\
0.0954044177197567	0.0395100985702848	0.00464387355632853	-0.00192426667289041\\
0.100048291276085	0.0375858318973944	0.0048532508794582	-0.00182285243570093\\
0.104901542155543	0.0357629794616934	0.0050693265942774	-0.00172686946823794\\
0.109970868749821	0.0340361099934555	0.0124051828319882	-0.00360665188023743\\
0.122376051581809	0.0304294581132181	0.0136403792111784	-0.00318919517530218\\
0.136016430792988	0.0272402629379159	0.0149464425274424	-0.00282080362816081\\
0.15096287332043	0.0244194593097551	0.0163156337520242	-0.00249643492365288\\
0.167278507072454	0.0219230243861022	0.0177379646904708	-0.00221082587722947\\
0.185016471762925	0.0197121985088727	0.0191991356774859	-0.00195865398775903\\
0.204215607440411	0.0177535445211137	0.0206802037771572	-0.00173635655361027\\
0.224895811217568	0.0160171879675034	0.0221598662879509	-0.00154075637579111\\
0.247055677505519	0.0144764315917123	0.0236148260658275	-0.00136862504061662\\
0.270670503571346	0.0131078065510957	0.0250168450404216	-0.00121682803384411\\
0.295687348611768	0.0118909785172516	0.0263365649737166	-0.00108313779768043\\
0.322023913585485	0.0108078407195712	0.0275439067714141	-0.000965546578641967\\
0.349567820356899	0.00984229414092919	0.0286094206110371	-0.000862068378690824\\
0.378177240967936	0.00898022576223837	0.0295031647468257	-0.000770845541134187\\
0.407680405714761	0.00820938022110418	0.0302020361590327	-0.000690485104252602\\
0.437882441873794	0.00751889511685158	0.0306860046494984	-0.000619727979648459\\
0.468568446523293	0.00689916713720312	0.030939090653901	-0.000557357123065147\\
0.499507537177194	0.00634181001413797	0.0309528794705013	-0.000502268506365559\\
0.530460416647695	0.00583954150777241	0.030729298981966	-0.000453604876216367\\
0.561189715629661	0.00538593663155604	0.0302752536120752	-0.000410595609653376\\
0.591464969241736	0.00497534102190267	0.0296020329941821	-0.000372507629365648\\
0.621067002235918	0.00460283339253702	0.0287305554611306	-0.000338690197141319\\
0.649797557697049	0.0042641431953957	0.0276859456698004	-0.000308631996600205\\
0.677483503366849	0.0039555111987955	0.0264959060281605	-0.000281878284592944\\
0.70397940939501	0.00367363291420255	0.0251893205690213	-0.000257999519807863\\
0.729168729964031	0.00341563339439469	0.0237976544079443	-0.000236618041836528\\
0.752966384371975	0.00317901535255816	0.0223493775214789	-0.000217439140901362\\
0.775315761893454	0.0029615762116568	0.0208722827704292	-0.000200203665638357\\
0.796188044663883	0.00276137254601844	0.0193923101477921	-0.000184666304179972\\
0.815580354811676	0.00257670624183847	0.0179305261472381	-0.000170610433876193\\
0.833510880958914	0.00240609580796228	0.0165041028396153	-0.000157869976911734\\
0.850014983798529	0.00224822583105054	0.0151286857867993	-0.000146299980173533\\
0.865143669585328	0.00210192585087701	0.0138172205183434	-0.000135761962560815\\
0.878960890103672	0.0019661638883162	0.0125756030512876	-0.000126131654623838\\
0.891536493154959	0.00184003223369236	0.0114088470560861	-0.000117315547015864\\
0.902945340211045	0.00172271668667649	0.0103213218928495	-0.000109232118553118\\
0.913266662103895	0.00161348456812338	0.00534949212292923	-5.80413262936733e-05\\
0.918616154226824	0.0015554432418297	0.00504592283972471	-5.58047483219342e-05\\
0.923662077066549	0.00149963849350777	0.00475621083858824	-5.36681390426621e-05\\
0.928418287905137	0.00144597035446511	0.00448020578253194	-5.16260228734019e-05\\
};
\addlegendentry{data18}

\addplot[-Straight Barb, color=mycolor5, point meta={sqrt((\thisrow{u})^2+(\thisrow{v})^2)}, point meta min=0, quiver={u=\thisrow{u}, v=\thisrow{v}, every arrow/.append style={-{Straight Barb[angle'=18.263, scale={10/1000*\pgfplotspointmetatransformed}]}}}]
 table[row sep=crcr] {%
x	y	u	v\\
0.0934147860265107	0.0356886221591872	0.00458633068447069	-0.0017450895195317\\
0.0980011167109814	0.0339435326396555	0.00479450476900736	-0.0016528573774177\\
0.102795621479989	0.0322906752622378	0.00500944414214791	-0.00156556568828801\\
0.107805065622137	0.0307251095739498	0.00523110493762012	-0.00148296282426845\\
0.113036170559757	0.0292421467496813	0.0127649260668555	-0.00309016991128347\\
0.125801096626612	0.0261519768383979	0.0140228130920478	-0.00273261548581502\\
0.13982390971866	0.0234193613525828	0.0153497562420047	-0.00241712486060659\\
0.155173665960665	0.0210022364919762	0.0167371297330106	-0.00213936474260074\\
0.171910795693675	0.0188628717493755	0.0181740110286178	-0.001894819282095\\
0.190084806722293	0.0169680524672805	0.0196450102775001	-0.00167893445577369\\
0.209729816999793	0.0152891180115068	0.0211301444044814	-0.0014886508389097\\
0.230859961404275	0.0138004671725971	0.0226070916119963	-0.00132123737419818\\
0.253467053016271	0.0124792297983989	0.0240516359681065	-0.00117392406288825\\
0.277518688984377	0.0113053057355107	0.0254347136512769	-0.0010440289762646\\
0.302953402635654	0.0102612767592461	0.0267266348913082	-0.000929638789461487\\
0.329680037526963	0.0093316379697846	0.027897223097272	-0.000829028080771807\\
0.357577260624235	0.0085026098890128	0.0289171968487646	-0.000740493162514266\\
0.386494457472999	0.00776211672649853	0.029757321571467	-0.000662444399798168\\
0.416251779044466	0.00709967232670036	0.0303957729891783	-0.000593686163479282\\
0.446647552033645	0.00650598616322108	0.030814076480234	-0.000533137876091159\\
0.477461628513879	0.00597284828712992	0.030997985385879	-0.000479756533340568\\
0.508459613899758	0.00549309175378935	0.0309413085600398	-0.000432597857889335\\
0.539400922459797	0.00506049389590002	0.0306480780069099	-0.000390927540059947\\
0.570049000466707	0.00466956635584007	0.0301272762394538	-0.000354085923583318\\
0.600176276706161	0.00431548043225675	0.0293921307212921	-0.000321446028924545\\
0.629568407427453	0.00399403440333221	0.0284652450181658	-0.00029245199119499\\
0.658033652445619	0.00370158241213722	0.0273728046171402	-0.00026666705750183\\
0.685406457062759	0.00343491535463539	0.0261433085623375	-0.000243703150802945\\
0.711549765625097	0.00319121220383244	0.02480615894139	-0.000223193767570199\\
0.736355924566487	0.00296801843626224	0.0233926773368534	-0.000204816753580801\\
0.75974860190334	0.00276320168268144	0.0219308858882473	-0.000188320996336561\\
0.781679487791587	0.00257488068634488	0.0204479701326155	-0.000173485854326366\\
0.802127457924203	0.00240139483201851	0.0189691169844212	-0.000160102374937364\\
0.821096574908624	0.00224129245708115	0.0175142691769854	-0.000147985905183246\\
0.838610844085609	0.0020933065518979	0.0160996207446054	-0.000136995099096355\\
0.854710464830215	0.00195631145280155	0.0147398655930391	-0.00012700663540468\\
0.869450330423254	0.00182930481739687	0.013446996251129	-0.000117902600342036\\
0.882897326674383	0.00171140221705483	0.012225957212801	-0.00010957702704723\\
0.895123283887184	0.0016018251900076	0.0110810583272759	-0.00010195027966638\\
0.90620434221446	0.00149987491034122	0.0100160458635974	-9.49529245522196e-05\\
0.916220388078057	0.001404921985789	0.00517914466543745	-5.03776705630277e-05\\
0.921399532743495	0.00135454431522598	0.00488367232250009	-4.84464651352358e-05\\
0.926283205065995	0.00130609785009074	0.00460190274137784	-4.66008710004464e-05\\
0.930885107807373	0.00125949697909029	0.00433365506082428	-4.48362448780409e-05\\
};
\addlegendentry{data19}

\addplot[-Straight Barb, color=mycolor6, point meta={sqrt((\thisrow{u})^2+(\thisrow{v})^2)}, point meta min=0, quiver={u=\thisrow{u}, v=\thisrow{v}, every arrow/.append style={-{Straight Barb[angle'=18.263, scale={10/1000*\pgfplotspointmetatransformed}]}}}]
 table[row sep=crcr] {%
x	y	u	v\\
0.0954902241444074	0.0296920375328275	0.00471526258314844	-0.0014517937352706\\
0.100205486727556	0.0282402437975569	0.00492828901648196	-0.001374918087616\\
0.105133775744038	0.0268653257099409	0.00514811082429509	-0.00130217119747576\\
0.110281886568333	0.0255631545124651	0.00537466535578766	-0.001233342215054\\
0.115656551924121	0.0243298122974111	0.013083729016363	-0.00256503986627626\\
0.128740280940484	0.0217647724311349	0.014361307438262	-0.00226831204767064\\
0.143101588378746	0.0194964603834642	0.0157062214466617	-0.00200652433929619\\
0.158807809825407	0.017489936044168	0.0171090510568391	-0.00177606948059911\\
0.175916860882246	0.0157138665635689	0.0185580419872249	-0.00157319326365454\\
0.194474902869471	0.0141406732999144	0.0200368381933392	-0.00139411871767428\\
0.21451174106281	0.0127465545822401	0.0215245530875073	-0.00123630075143667\\
0.236036294150318	0.0115102538308034	0.0229979895251125	-0.00109746554049235\\
0.25903428367543	0.0104127882903111	0.0244321529856825	-0.000975310411064006\\
0.283466436661113	0.00943747787924708	0.0257972999831169	-0.000867611213277907\\
0.30926373664423	0.00856986666596917	0.02706351364858	-0.00077277606976745\\
0.33632725029281	0.00779709059620172	0.0282006049316771	-0.000689368836706013\\
0.364527855224487	0.00710772175949571	0.0291795147758044	-0.00061597360433503\\
0.393707370000291	0.00649174815516068	0.0299717148020657	-0.000551272060142252\\
0.423679084802357	0.00594047609501843	0.0305565687920236	-0.000494270159160544\\
0.45423565359438	0.00544620593585789	0.0309170243486153	-0.000444069647257111\\
0.485152677942996	0.00500213628860078	0.031040397613415	-0.000399804549969857\\
0.516193075556411	0.00460233173863092	0.0309224593759965	-0.000360692193013225\\
0.547115534932407	0.00424163954561769	0.0305690614825871	-0.000326123224390029\\
0.577684596414994	0.00391551632122766	0.0299909664542701	-0.000295550579144401\\
0.607675562869264	0.00361996574208326	0.0292030447517161	-0.000268454661854526\\
0.636878607620981	0.00335151108022874	0.0282292833788864	-0.000244375296387103\\
0.665107890999867	0.00310713578384163	0.0270967105541019	-0.00022295096293612\\
0.692204601553969	0.00288418482090551	0.0258344373256267	-0.000203860690242547\\
0.718039038879596	0.00268032413066297	0.0244722371284967	-0.000186801388099428\\
0.742511276008092	0.00249352274256354	0.0230412280101823	-0.000171506691721661\\
0.765552504018275	0.00232201605084188	0.0215689940033256	-0.000157769192111487\\
0.7871214980216	0.00216424685873039	0.020082155318252	-0.000145406771445045\\
0.807203653339852	0.00201884008728535	0.0186052151239466	-0.000134246888325988\\
0.825808868463799	0.00188459319895936	0.0171571304650033	-0.00012413696367838\\
0.842965998928802	0.00176045623528098	0.0157532514361336	-0.000114960365475105\\
0.858719250364936	0.00164549586980587	0.0144074574822129	-0.000106615366359377\\
0.873126707847149	0.0015388805034465	0.0131309289028012	-9.90046278986721e-05\\
0.88625763674995	0.00143987587554782	0.0119278091196393	-9.20405621571564e-05\\
0.898185445869589	0.00134783531339067	0.0108018204469246	-8.56574177120399e-05\\
0.908987266316514	0.00126217789567863	0.00975619131029803	-7.97978607369187e-05\\
0.918743457626812	0.00118238003494171	0.00503468769067184	-4.22828012880481e-05\\
0.923778145317484	0.00114009723365366	0.00474609837518081	-4.06690621419143e-05\\
0.928524243692664	0.00109942817151175	0.00447107217788634	-3.91263583395154e-05\\
0.932995315870551	0.00106030181317223	0.00420940372554079	-3.7650869760277e-05\\
};
\addlegendentry{data20}

\addplot[-Straight Barb, color=mycolor7, point meta={sqrt((\thisrow{u})^2+(\thisrow{v})^2)}, point meta min=0, quiver={u=\thisrow{u}, v=\thisrow{v}, every arrow/.append style={-{Straight Barb[angle'=18.263, scale={10/1000*\pgfplotspointmetatransformed}]}}}]
 table[row sep=crcr] {%
x	y	u	v\\
0.0971811568323542	0.0235758935509427	0.00482726413535335	-0.00115269143613786\\
0.102008420967708	0.0224232021148049	0.00504447327203443	-0.00109155390276394\\
0.107052894239742	0.0213316482120409	0.00526849799963744	-0.00103370744183861\\
0.112321392239379	0.0202979407702023	0.0054992588536987	-0.000978983591553136\\
0.117820651093078	0.0193189571786492	0.0133606188844425	-0.00203281505731443\\
0.131181269977521	0.0172861421213348	0.0146550902653608	-0.001797678862258\\
0.145836360242881	0.0154884632590768	0.0160153334593296	-0.00159025518794206\\
0.161851693702211	0.0138982080711347	0.0174312337475696	-0.00140767748116049\\
0.179282927449781	0.0124905305899742	0.018890312714165	-0.00124696561545295\\
0.198173240163946	0.0112435649745213	0.0203753750963442	-0.00110512728589708\\
0.21854861526029	0.0101384376886242	0.0218647621046635	-0.000980141237137825\\
0.240413377364953	0.00915829645148636	0.0233345356926047	-0.00087020038324237\\
0.263747913057558	0.00828809606824399	0.0247590495099662	-0.000773477046256226\\
0.288506962567524	0.00751461902198776	0.0261080043232559	-0.000688209392941264\\
0.31461496689078	0.00682640962904649	0.0273513299868909	-0.000613133123958809\\
0.341966296877671	0.00621327650508769	0.0284588753753376	-0.000547107244342899\\
0.370425172253009	0.00566616926074479	0.0294018218766213	-0.000489008250205349\\
0.39982699412963	0.00517716101053944	0.0301523038177817	-0.000437791954661944\\
0.429979297947412	0.00473936905587749	0.0306907489209211	-0.000392669425533333\\
0.460670046868333	0.00434669963034416	0.0310013642403839	-0.000352927867618244\\
0.491671411108717	0.00399377176272592	0.0310728356555043	-0.000317880791227837\\
0.522744246764221	0.00367589097149808	0.030902624040134	-0.000286908632850937\\
0.553646870804355	0.00338898233864714	0.0304981259724911	-0.000259528601092614\\
0.584144996776846	0.00312945373755453	0.0298716037590663	-0.000235307492049901\\
0.614016600535912	0.00289414624550463	0.0290393006974151	-0.000213834095171735\\
0.643055901233327	0.00268031215033289	0.0280263282118545	-0.000194744546679856\\
0.671082229445182	0.00248556760365304	0.0268603817897699	-0.00017775304654625\\
0.697942611234952	0.00230781455710679	0.0255710432217161	-0.000162606027087296\\
0.723513654456668	0.00214520853001949	0.0241883525582823	-0.000149064067475616\\
0.74770200701495	0.00199614446254388	0.0227432046171778	-0.000136916787260158\\
0.770445211632128	0.00185922767528372	0.0212627811238524	-0.000126000525499615\\
0.79170799275598	0.0017332271497841	0.0197731994440005	-0.00011617166115341\\
0.811481192199981	0.00161705548863069	0.018298363953432	-0.000107294075106569\\
0.829779556153413	0.00150976141352412	0.016856387858192	-9.92473258164901e-05\\
0.846635944011605	0.00141051408770763	0.0154619082335385	-9.19394592668696e-05\\
0.862097852245143	0.00131857462844076	0.0141281208445049	-8.52902933516511e-05\\
0.876225973089648	0.00123328433508911	0.0128655258115	-7.92230566480292e-05\\
0.889091498901148	0.00115406127844108	0.0116776008976502	-7.36685925015535e-05\\
0.900769099798799	0.00108039268593953	0.0105675838597482	-6.85750347436409e-05\\
0.911336683658547	0.00101181765119589	0.00953827759361292	-6.38971531281326e-05\\
0.92087496125216	0.000947920498067755	0.00491403895659348	-3.38221422577791e-05\\
0.925789000208753	0.000914098355809976	0.00463117791013046	-3.25360560735156e-05\\
0.930420178118884	0.00088156229973646	0.00436176051695303	-3.13062486756061e-05\\
0.934781938635837	0.000850256051060854	0.00410556174204491	-3.01297151636132e-05\\
};
\addlegendentry{data21}

\addplot[-Straight Barb, color=mycolor1, point meta={sqrt((\thisrow{u})^2+(\thisrow{v})^2)}, point meta min=0, quiver={u=\thisrow{u}, v=\thisrow{v}, every arrow/.append style={-{Straight Barb[angle'=18.263, scale={10/1000*\pgfplotspointmetatransformed}]}}}]
 table[row sep=crcr] {%
x	y	u	v\\
0.0984807753012208	0.0173648177666931	0.0049216152006834	-0.000848981483305644\\
0.103402390501904	0.0165158362833874	0.00514233277246233	-0.000803890158331683\\
0.108544723274367	0.0157119461250557	0.00536987934222745	-0.00076123085413592\\
0.113914602616594	0.0149507152709198	0.00560416100640584	-0.000720878837556366\\
0.119518763623	0.0142298364333635	0.0135945728802174	-0.00149503791704628\\
0.133113336503217	0.0127347985163172	0.0149033032225922	-0.0013221020342773\\
0.148016639725809	0.0114126964820399	0.0162764566701523	-0.00116956561271526\\
0.164293096395962	0.0102431308693246	0.0177033290848997	-0.0010353150858277\\
0.181996425480861	0.00920781578349691	0.0191708296497426	-0.000917154933616516\\
0.201167255130604	0.0082906608498804	0.0206610530479839	-0.000812884119269357\\
0.221828308178588	0.00747777673061104	0.022151694397971	-0.000721013131934197\\
0.243980002576559	0.00675676359867684	0.0236182004505066	-0.000640209521551619\\
0.267598203027066	0.00611655407712523	0.025034387522512	-0.000569127016103708\\
0.292632590549578	0.00554742706102152	0.0263695046411254	-0.000506469943521171\\
0.319002095190703	0.00504095711750035	0.0275933755723218	-0.000451306623534477\\
0.346595470763025	0.00458965049396587	0.0286759076577358	-0.000402796017866001\\
0.375271378420761	0.00418685447609987	0.0295885102744113	-0.00036011082221898\\
0.404859888695172	0.00382674365388089	0.0303039015412607	-0.000322483271343231\\
0.435163790236433	0.00350426038253766	0.0308034221561164	-0.00028933237768524\\
0.465967212392549	0.00321492800485242	0.031072352797905	-0.000260133300555738\\
0.497039565190454	0.00295479470429668	0.0311005386317402	-0.000234380981453359\\
0.528140103822194	0.00272041372284332	0.0308868529586578	-0.000211620143668947\\
0.559026956780852	0.00250879357917438	0.0304399717070398	-0.000191495783850583\\
0.589466928487892	0.00231729779532379	0.0297733934552659	-0.000173689511355946\\
0.619240321943158	0.00214360828396785	0.0289044861983284	-0.000157899243473523\\
0.648144808141486	0.00198570904049432	0.0278592630755186	-0.000143857876011388\\
0.676004071217005	0.00184185116448294	0.0266659462709352	-0.000131355656941536\\
0.70267001748794	0.0017104955075414	0.0253544820187939	-0.000120206600048219\\
0.728024499506734	0.00159028890749318	0.023955105741245	-0.00011023510642545\\
0.751979605247979	0.00148005380106773	0.0224985005419671	-0.000101286894349323\\
0.774478105789946	0.00137876690671841	0.0210115012403488	-9.32420437362801e-05\\
0.795489607030295	0.00128552486298213	0.0195197998551073	-8.59953699652723e-05\\
0.815009406885402	0.00119952949301686	0.0180467980826682	-7.94471513681095e-05\\
0.83305620496807	0.00112008234164875	0.0166099088320947	-7.35091087044897e-05\\
0.849666113800165	0.00104657323294426	0.0152231836692918	-6.81139133367251e-05\\
0.864889297469457	0.000978459319607532	0.0138992596526083	-6.32028689715604e-05\\
0.878788557122065	0.000915256450635971	0.0126480812769931	-5.87197462712273e-05\\
0.891436638399058	0.000856536704364744	0.0114725838816382	-5.46138473711453e-05\\
0.902909222280696	0.000801922856993599	0.0103756130523319	-5.08471804294145e-05\\
0.913284835333028	0.000751075676564184	0.00935962827891534	-4.73866120761199e-05\\
0.922644463611944	0.000703689064488064	0.00481559432987233	-2.50628336715135e-05\\
0.927460057941816	0.000678626230816551	0.00453735451368642	-2.41126507945871e-05\\
0.931997412455502	0.000654513580021964	0.00427246289402983	-2.3203848285296e-05\\
0.936269875349532	0.000631309731736667	0.00402067874122858	-2.23342296406417e-05\\
};
\addlegendentry{data22}

\addplot[-Straight Barb, color=mycolor2, point meta={sqrt((\thisrow{u})^2+(\thisrow{v})^2)}, point meta min=0, quiver={u=\thisrow{u}, v=\thisrow{v}, every arrow/.append style={-{Straight Barb[angle'=18.263, scale={10/1000*\pgfplotspointmetatransformed}]}}}]
 table[row sep=crcr] {%
x	y	u	v\\
0.0993838464461254	0.0110838199901011	0.00499763120481068	-0.000541880756486443\\
0.104381477650936	0.0105419392336147	0.00522117647599928	-0.000513068016350078\\
0.109602654126935	0.0100288712172646	0.00545155999938714	-0.000485811753477646\\
0.115054214126323	0.00954305946378694	0.00568867616063409	-0.00046003201371185\\
0.120742890286957	0.00908302745007509	0.0137845167390935	-0.000953240778390431\\
0.13452740702605	0.00812978667168466	0.0151049961877786	-0.000842959793575326\\
0.149632403213829	0.00728682687810934	0.0164888150650849	-0.000745698275795871\\
0.166121218278914	0.00654112860231347	0.0179247902186102	-0.00066010545574077\\
0.184046008497524	0.0058810231465727	0.0193993337649994	-0.00058477896771148\\
0.203445342262523	0.00529624417886122	0.0208939614205732	-0.000518314574168901\\
0.224339303683096	0.00477792960469232	0.022385843864157	-0.000459760778314403\\
0.246725147547253	0.00431816882637791	0.023849930831233	-0.000408266024489134\\
0.270575078378486	0.00390990280188878	0.0252596053557414	-0.000362970568033615\\
0.295834683734228	0.00354693223385516	0.026583751787805	-0.000323048012265586\\
0.322418435522033	0.00322388422158958	0.0277921126182458	-0.000287903238831406\\
0.350210548140279	0.00293598098275817	0.0288546492445158	-0.000256998917160619\\
0.379065197384794	0.00267898206559755	0.0297429612131202	-0.000229806904033741\\
0.408808158597915	0.00244917516156381	0.0304302424766154	-0.000205837636692448\\
0.43923840107453	0.00224333752487136	0.0308985741721139	-0.000184720228703669\\
0.470136975246644	0.00205861729616769	0.0311341065119507	-0.00016611967960668\\
0.501271081758595	0.00189249761656101	0.0311276199858515	-0.000149713822088003\\
0.532398701744446	0.00174278379447301	0.0308791191115332	-0.000135212522527672\\
0.563277820855979	0.00160757127194534	0.0303983024707744	-0.000122389447230855\\
0.593676123326754	0.00148518182471448	0.0296996556014352	-0.00011104168439385\\
0.623375778928189	0.00137414014032063	0.0288014413901018	-0.000100976787617752\\
0.652177220318291	0.00127316335270288	0.0277303815213491	-9.20246966238929e-05\\
0.67990760183964	0.00118113865607899	0.0265151120144049	-8.40518973390535e-05\\
0.706422713854045	0.00109708675873994	0.0251858647494442	-7.69400611232139e-05\\
0.731608578603489	0.00102014669761672	0.0237730254630463	-7.0577473361351e-05\\
0.755381604066535	0.000949569224255371	0.0223071010377301	-6.48659918551903e-05\\
0.777688705104265	0.000884703232400181	0.0208146495546073	-5.97293867370019e-05\\
0.798503354658873	0.000824973845663179	0.0193210263020789	-5.51008221353186e-05\\
0.817824380960951	0.00076987302352786	0.0178492345615668	-5.09169174883299e-05\\
0.835673615522518	0.00071895610603953	0.0164161312869406	-4.71215436401543e-05\\
0.852089746809459	0.000671834562399376	0.0150353070654644	-4.36719293595288e-05\\
0.867125053874923	0.000628162633039847	0.0137189604644066	-4.05307942310776e-05\\
0.88084401433933	0.00058763183880877	0.0124765989184153	-3.76624095684634e-05\\
0.893320613257745	0.000549969429240306	0.0113107291320262	-3.50345385651862e-05\\
0.904631342389771	0.00051493489067512	0.0102238887665907	-3.26230399626371e-05\\
0.914855231156362	0.000482311850712483	0.00921826824648497	-3.04068587872621e-05\\
0.924073499402847	0.000451904991925221	0.00473816411305916	-1.60735364465856e-05\\
0.928811663515906	0.000435831455478635	0.00446347468057218	-1.54655457994428e-05\\
0.933275138196479	0.000420365909679192	0.00420206478617202	-1.48839345024348e-05\\
0.937477202982651	0.000405481975176757	0.00395368131339668	-1.43273086912626e-05\\
};
\addlegendentry{data23}

\addplot[-Straight Barb, color=mycolor3, point meta={sqrt((\thisrow{u})^2+(\thisrow{v})^2)}, point meta min=0, quiver={u=\thisrow{u}, v=\thisrow{v}, every arrow/.append style={-{Straight Barb[angle'=18.263, scale={10/1000*\pgfplotspointmetatransformed}]}}}]
 table[row sep=crcr] {%
x	y	u	v\\
0.0998867339183008	0.00475819158237424	0.00505466975334547	-0.000232619305406283\\
0.104941403671646	0.00452557227696796	0.00528035325713133	-0.00022024001222472\\
0.110221756928778	0.00430533226474324	0.0055128822896804	-0.000208530250509043\\
0.115734639218458	0.0040968020142342	0.00575214255323561	-0.000197455586925573\\
0.121486781771694	0.00389934642730862	0.0139293262449629	-0.000408946926179019\\
0.135416108016656	0.0034903995011296	0.0152591252116851	-0.000361622732042214\\
0.150675233228342	0.00312877676908739	0.0166514863583732	-0.000319889828461294\\
0.167326719586715	0.0028088869406261	0.0180948625098722	-0.000283167500554601\\
0.185421582096587	0.00252571944007149	0.0195752874413867	-0.000250853032065837\\
0.204996869537974	0.00227486640800566	0.0210738310885505	-0.000222343376160225\\
0.226070700626524	0.00205252303184543	0.0225672581599777	-0.00019722962978629\\
0.248637958786502	0.00185529340205914	0.024030133852159	-0.00017514571330518\\
0.272668092638661	0.00168014768875396	0.025435503864625	-0.000155722207905139\\
0.298103596503286	0.00152442548084882	0.0267519615287918	-0.000138604339218567\\
0.324855558032078	0.00138582114163026	0.0279491749768114	-0.000123536382414617\\
0.352804733008889	0.00126228475921564	0.0289971352334926	-0.000110287408520407\\
0.381801868242382	0.00115199735069523	0.0298675738495144	-9.86305621955932e-05\\
0.411669442091896	0.00105336678849964	0.0305340293832144	-8.83557413529709e-05\\
0.44220347147511	0.00096501104714667	0.0309791335236361	-7.93036573294783e-05\\
0.473182604998747	0.000885707389817191	0.0311896871933767	-7.13304484849316e-05\\
0.504372292192123	0.00081437694133226	0.0311571712877459	-6.42978292828382e-05\\
0.535529463479869	0.000750079112049422	0.030882437355361	-5.80813557896202e-05\\
0.56641190083523	0.000691997756259801	0.0303759563922683	-5.25839232651129e-05\\
0.596787857227498	0.000639413832994689	0.0296529622986018	-4.77185022696785e-05\\
0.6264408195261	0.00059169533072501	0.0287323969716069	-4.34025754780044e-05\\
0.655173216497707	0.000548292755247006	0.027641520094326	-3.95632592729097e-05\\
0.682814736592033	0.000508729495974096	0.0264092896881242	-3.61433455712605e-05\\
0.709224026280157	0.000472586150402835	0.0250661652889891	-3.30921523462211e-05\\
0.734290191569146	0.000439493998056614	0.0236426577116717	-3.03618324985423e-05\\
0.757932849280818	0.000409132165558072	0.0221691512601516	-2.79103566184417e-05\\
0.78010200054097	0.00038122180893963	0.020672008573526	-2.57050979950607e-05\\
0.800774009114496	0.000355516710944569	0.019176345184954	-2.37174598482799e-05\\
0.81995035429945	0.00033179925109629	0.017704876398996	-2.19203188339377e-05\\
0.837655230698446	0.000309878932262352	0.0162740482901855	-2.0289648660821e-05\\
0.853929278988631	0.000289589283601531	0.0148971128917627	-1.88071536929899e-05\\
0.868826391880394	0.000270782129908541	0.0135859470712806	-1.74568895979032e-05\\
0.882412338951674	0.000253325240310638	0.0123497354572794	-1.62235708724443e-05\\
0.894762074408954	0.000237101669438194	0.0111906630764546	-1.50933988487536e-05\\
0.905952737485408	0.00022200827058944	0.0101110380139952	-1.40560457708292e-05\\
0.916063775499404	0.000207952224818611	0.00911285033619003	-1.31025040709605e-05\\
0.925176625835594	0.00019484972074765	0.00468092783244256	-6.92423593061584e-06\\
0.929857553668036	0.000187925484817034	0.00440874261025737	-6.6627426679014e-06\\
0.934266296278293	0.000181262742149133	0.00414979711627117	-6.41256475336492e-06\\
0.938416093394565	0.000174850177395768	0.00390382868061334	-6.17310623683242e-06\\
};
\addlegendentry{data24}

\addplot [color=red, draw=none, mark size=3.3pt, mark=*, mark options={solid, red}]
  table[row sep=crcr]{%
0	0\\
};
\addlegendentry{data25}

\end{axis}

\begin{axis}[%
width=5.931in,
height=4.297in,
at={(0in,0in)},
scale only axis,
xmin=0,
xmax=1,
ymin=0,
ymax=1,
axis line style={draw=none},
ticks=none,
axis x line*=bottom,
axis y line*=left,
legend style={legend cell align=left, align=left, draw=white!15!black}
]
\end{axis}
\end{tikzpicture}%
        \end{figure}
%\caption{Эллипсоидальные аппроксимации для 10 направлений.}
	Найдём собственные значения матрицы Якоби $J(u,v)$ для точки $P_2 = \left(\frac{B}{A},\,0\right)$. Выпишем характеристический многочлен:
	$$
                \chi_2=\mathrm{det}\begin{pmatrix}
                        -B - \lambda & -\frac{B}{A+B} \\
                        0 & -C + \frac{B}{A+B} - \lambda
                \end{pmatrix}
                =
                (-B - \lambda)\left(
                        -C + \frac{B}{A+B} - \lambda
                \right)
        $$
        Приравняв нулю характеристический многочлен $\chi_1(\lambda)$, получили собственные значения:
        $$
               \lambda_1 = -B, \quad \lambda_2 = -C + \frac{B}{A+B}.     
        $$
        Заметим, что при любом значении параметров собственные числа $\lambda_1$ и $\lambda_2$ вещественные; $\lambda_1 < 0$ при любом значении параметров, а $\lambda_2 > 0$ при $C < \frac{B}{A+B}$, $\lambda_2 < 0$ при $C > \frac{B}{A+B}$ и $\lambda_2 = 0$ при $C = \frac{B}{A+B}$.

        Таким образом,
        \begin{enumerate}
                \item При $C > \frac{B}{A+B}$ $P_1$ является устойчивым узлом;
                \item При $C < \frac{B}{A+B}$ $P_1$ является седлом.
        \end{enumerate}

        Найдём собственные значения матрицы Якоби $J(u,v)$ для точки $P_3=\left(\frac{C}{1 - C},\, \frac{B - BC - AC}{(1 - C)^2}\right)$. Выпишем характеристический многочлен:
        \begin{multline*}
                \chi_3 = \mathrm{det}\begin{pmatrix}
                        \frac{C}{1-C}(-A + B - (A+B) C) - \lambda
                        &
                        -C
                        \\
                        B - (A+B)C
                        &
                        -\lambda
                \end{pmatrix}
                =\\=
                \lambda\left(\frac{C}{1-C}(-A + B - (A+B)C) - \lambda\right) + C(B - (A+B)C) 
                =\\=
                \lambda^2 - \frac{C}{1-C}(-A + B - (A + B)C)\lambda + C(B - (A+B)C).
        \end{multline*}
        В таком случае:
        $$
                \lambda_{1,2} = \frac{\frac{C}{1-C}(-A + B - (A+B)C) \pm \sqrt{\frac{C^2}{(1-C)^2}(-A+B-(A+B)C)^2 - 4(B-(A+B)C)}}{2}
        $$
        Введём обозначения: $M = \frac{C}{1-C}(-A + B - (A+B)C)$, $N = 4(B-(A+B)C)$ и заметим, что $N$ на рассматриваемом промежутке $0 < C \leqslant \frac{B}{A+B}$ ограничена $0 \leqslant N < 4B$. Таким образом устойчивость выражения не зависит от подкоренного выражения.
        $$
                \lambda_{1,2} = \frac{M \pm \sqrt{M^2 - N}}{2}, \quad \mbox{где } N\geqslant0.
        $$
        Получается, что точка устойчива при $\frac{B-A}{A+B} < C < \frac{B}{A+B}$ и неустойчива при $0 < C < \frac{B-A}{A+B}$. При этом для анализа характера точек нужно исследовать знак подкоренного выражения, что сложно. Мы будем смотреть на графики, видно только, что системе характерны и как фокусы так и узлы. 
\end{document}